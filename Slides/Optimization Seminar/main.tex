\documentclass[10pt]{beamer}
%\usepackage[utf8]{inputenc}
%\usepackage{xeCJK}
\usepackage{graphicx}
\usepackage{subcaption}
\usepackage {mathtools}
%\usepackage{utopia} %font utopia imported
\usetheme{CambridgeUS}
\usecolortheme{dolphin}

% set colors
\definecolor{myNewColorA}{RGB}{25,25,112}
\definecolor{myNewColorB}{RGB}{25,25,112}
\definecolor{myNewColorC}{RGB}{25,25,112}
\setbeamercolor*{palette primary}{bg=myNewColorC}
\setbeamercolor*{palette secondary}{bg=myNewColorB, fg = white}
\setbeamercolor*{palette tertiary}{bg=myNewColorA, fg = white}
\setbeamercolor*{titlelike}{fg=myNewColorA}
\setbeamercolor*{title}{bg=myNewColorA, fg = white}
\setbeamercolor*{item}{fg=myNewColorA}
\setbeamercolor*{caption name}{fg=myNewColorA}
\usefonttheme{professionalfonts}
\usepackage{natbib}
\usepackage{hyperref}
%------------------------------------------------------------
%\titlegraphic{\includegraphics[height=1.5cm]{logo.png}} 

\setbeamerfont{title}{size=\large}
\setbeamerfont{subtitle}{size=\small}
\setbeamerfont{author}{size=\small}
\setbeamerfont{date}{size=\small}
\setbeamerfont{institute}{size=\small}



%设置页码脚标
%\setbeamertemplate{footline}[frame number]
\defbeamertemplate{footline}{NGEGFootlineTemplate}{%
	\leavevmode% 离开vmode,也就是离开竖直模式,进入水平模式
	\hbox{%
		\begin{beamercolorbox}[wd=.2\paperwidth,ht=2.25ex,dp=1ex,center]{author in head/foot}%
			\ifnum \the\value{page}>1 \usebeamerfont{author in head/foot}\insertshortauthor\fi
		\end{beamercolorbox}%
		\begin{beamercolorbox}[wd=.6\paperwidth,ht=2.25ex,dp=1ex,center]{title in head/foot}%
			\ifnum \the\value{page}>1 \usebeamerfont{title in head/foot}\insertshorttitle\fi
		\end{beamercolorbox}%
		\begin{beamercolorbox}[wd=0.2\paperwidth,ht=2.25ex,dp=1ex,center]{author in head/foot}%
			\ifnum \the\value{page}>1 \insertframenumber{} / \inserttotalframenumber\fi
	\end{beamercolorbox}}%
%	\vskip0pt%
}
\setbeamertemplate{footline}[NGEGFootlineTemplate]



\newcommand{\mb}[1]{\mathbf{#1}}
\newcommand{\mc}[1]{\mathcal{#1}}
\newcommand{\E}{\mathcal{E}}
\newcommand{\C}{\mathcal{C}}
\newcommand{\x}{\mathbf{x}}
\newcommand{\dd}{\mathrm{d}}



\title[Proximal Version of Gradient-Based Optimization]{Some Theoretical Studies on Proximal Gradient Descent and Its Acceleration}%主标题
%\subtitle{ }%%副标题
\author[Bowen Li]{Bowen Li
\and joint work with Bin Shi and Ya-xiang Yuan}%%作者

\institute[LSEC]{Institute of Computational Mathematics and Scientific/Engineering Computing,\\
Academy of Mathematics and Systems Science,\\
Chinese Academy of Sciences}
\date[\textcolor{white} ]
{December 13, 2022}

%------------------------------------------------------------
%This block of commands puts the table of contents at the 
%beginning of each section and highlights the current section:
%\AtBeginSection[]
%{
%  \begin{frame}
%    \frametitle{Contents}
%    \tableofcontents[currentsection]
%  \end{frame}
%}
\AtBeginSection[]{
  \begin{frame}
  \vfill
  \centering
  \begin{beamercolorbox}[sep=8pt,center,shadow=true,rounded=true]{title}
    \usebeamerfont{title}\insertsectionhead\par%
  \end{beamercolorbox}
  \vfill
  \end{frame}
}
%------------------------------------------------------------

\begin{document}

%The next statement creates the title page.
\frame{\titlepage}
\begin{frame}
\frametitle{Contents}
\tableofcontents
\end{frame}
%------------------------------------------------------------
\section{Background}
\begin{frame}{Unconstraint convex optimization}
  The problem formulation is
  \begin{equation}\label{pro}
      \min_x f(x),
  \end{equation}
  where $x\in \mathbb{R}^n$ and $f(x)\in \mathcal{F}^1_L(\mathbb{R}^n)$, i.e.,
  \begin{equation*}
    f(x) \ge f(y) + \left\langle \nabla f(y), x - y\right\rangle, \quad \left\|\nabla f(x) - \nabla f(y)\right\| \le L \| x - y \|.
  \end{equation*}

  Gradient descent method (GD)
  \begin{equation*}
      x_{k+1} = x_k - s \nabla f(x_k),
  \end{equation*}
 where $s>0$ is the step size. This method can be traced back to~\citet{cauchy1847methode}.
\end{frame}

\begin{frame}{Nesterov's acceleration}
  The global convergence rate for GD 
  \begin{equation*}
    f(x_k) - f(x^\star) \le O\left( \frac{1}{sk}\right).
  \end{equation*}
  \pause
  The Nesterov's acceleration (NAG), ~\citet{nesterov1983method} 
  \begin{equation}\label{NAG}
    \left\{\begin{aligned}
      & x_{k+1} = y_{k} - s\nabla f(y_{k}),              \\
      & y_{k+1} = x_{k+1}  + \frac{k}{k+r+1}(x_{k+1} - x_{k}),
      \end{aligned}\right.
  \end{equation}
  where $r \ge 2$.
\end{frame}

\begin{frame}{Counterpart continuous dynamics}
  In~\citet{su2016differential}, it is found that NAG (with $r=2$) has relation to
  \begin{equation*}
    \ddot{X} + \frac{3}{t}\dot{X} + \nabla f(X) = 0.
  \end{equation*}
  \pause
  This idea is developed by~\citet{shi2021understanding}
  \begin{equation*}
    \ddot{X} + \frac{3}{t}\dot{X} + \sqrt{s}\nabla^2f(X)\dot{X} + \left( 1 + \frac{3\sqrt{s}}{2t} \right)\nabla f(X) = 0.
  \end{equation*}
\end{frame}

\begin{frame}
  In~\citet{chen2022gradient}, a general simplified framework is proposed. \par 
  
  Construct the Lyapunov function
  \begin{equation*}
    \mathcal{E}(t) = t\left(t+\frac{\sqrt{s}}{2}\right)(f(X) - f(x^\star)) + \frac{1}{2}\left\|t\dot{X} + 2(X - x^\star) + t\sqrt{s}\nabla f(X)\right\|^2.
  \end{equation*}

  The convergence rate for NAG with $r > 2$ is
  \begin{equation*}
    \min_{0\le i \le k}f(x_i) - f(x^\star) \le o\left( \frac{1}{k^2} \right) \quad \min_{0\le i \le k}\left\| \nabla f(x_i) \right\|^2 \le o\left( \frac{1}{k^3} \right).
  \end{equation*}
\end{frame}

\begin{frame}{Composite optimization}
  Consider the following problem which arises often in learning problems
  \begin{equation}\label{compro}
    \min_x \Phi(x):= f(x) + g(x),
  \end{equation}
  where $f(x)$ is L-smooth and convex and $g(x)$ is only convex. \par
  \pause
  One model for image deblurring problem
  \begin{equation*}
    \min_x \|Ax - b\|^2 + \lambda\|x\|_1,
  \end{equation*}
  where $A\in\mathbb{R}^{m\times n}$ is the blur operator, $b\in\mathbb{R}^m$ represents the blurred image.
\end{frame}

\begin{frame}{Illustration}
  \begin{figure}[htpb!]
    \centering
    \begin{subfigure}[t]{0.48\linewidth}
    \centering
    \includegraphics[scale=0.39]{fig/original.eps}
    \caption{Original}
    \end{subfigure}
    \begin{subfigure}[t]{0.48\linewidth}
    \centering
    \includegraphics[scale=0.39]{fig/corrupted.eps}
    \caption{Blurred and Noisy}
    \end{subfigure}
    \caption{The image deblurring problem: An image of an elephant}
    \label{fig: deblurring}
    \end{figure}
\end{frame}


\begin{frame}{Proximal gradient descent (ISTA)}
  

  The proximal gradient descent method is 
  \begin{equation}\label{nonacc}
    x_{k+1} = P_s(x_k),
  \end{equation} 
  where $P_s$ is the proximal operator 
  \begin{equation*}
    P_s(x) = \mathop{\arg\min}_z \left\{\left\| z - \left( x - s\nabla f(x) \right) \right\|^2 + g(z) \right\}.
  \end{equation*}
\end{frame}

\begin{frame}{Acceleratd proximal gradient descent (FISTA)}
  The correspondent accelerated proximal gradient descent method is
  \begin{equation*}
    \left\{\begin{aligned}
      & x_{k+1} = P_s(y_k),              \\
      & y_{k+1} = x_{k+1}  + \frac{k}{k+r+1}(x_{k+1} - x_{k}).
      \end{aligned}\right.
  \end{equation*}
Further, defined in~\citet{su2016differential}, the proximal subgradient
\begin{equation*}
  G_s(y):= \frac{y - P_s(y)}{s}
\end{equation*}

yields the rewritten scheme formation as 
\begin{equation}\label{acc}
  \left\{\begin{aligned}
    & x_{k+1} = y_k - sG_s(y_k),              \\
    & y_{k+1} = x_{k+1}  + \frac{k}{k+r+1}(x_{k+1} - x_{k}).
    \end{aligned}\right.
\end{equation}
\end{frame}

\begin{frame}{Illustration}
  \begin{figure}[htpb!]
    \centering
    \begin{subfigure}[t]{0.48\linewidth}
    \centering
    \includegraphics[scale=0.39]{fig/ista.eps}
    \caption{ISTA}
    \end{subfigure}
    \begin{subfigure}[t]{0.48\linewidth}
    \centering
    \includegraphics[scale=0.39]{fig/fista.eps}
    \caption{FISTA}
    \end{subfigure}
    \caption{Deblurred elephant's image by ISTA and FISTA.} 
    \end{figure}
\end{frame}


\section{Pillar inequality of analysis}
\begin{frame}{Key inequality}
	 In the convergence rate proof of NAG method for
   \begin{equation*}
    \min_x f(x)
   \end{equation*}
  
   the pivotal inequalities are 
   \begin{equation*}
    f(x) \ge f(y) + \left\langle \nabla f(y), x - y\right\rangle,
   \end{equation*}
   and 
   \begin{equation*}
    f(x) \le f(y) + \left\langle \nabla f(y), x - y\right\rangle - \frac{1}{2L}\left\| \nabla f(x) - \nabla f(y) \right\|^2.
   \end{equation*}
\end{frame}

\begin{frame}{Key inequality}
  For composite optimization,In~\citep[Lemma 2.3]{beck2009fast}, they derives an inequality 
  \begin{equation*}
    \Phi(y - sG_s(y)) - \Phi(x) \le \left\langle G_s(y), x - y\right\rangle - \frac{s^2}{2}\|G_s(y)\|^2,
  \end{equation*}
  where $s \le 1/L$. We tight this inequality to 
  \begin{equation}\label{ine}
    \Phi(y - sG_s(y)) \le \Phi(x) + \langle G_s(y), y - x\rangle - \left(s - \frac{s^2 L}{2}\right)\|G_s(y)\|^2,
  \end{equation}
  where $s < 1/L$. This improvement is essencial to get the subgradient norm minimization for FISTA.
\end{frame}

\begin{frame}
  \begin{theorem}[Convergence rate for nonaccelerated case]
    \label{thm: ista}
    Let $\Phi = f + g$ be a composite function with $f \in \mathcal{F}_{L}^1(\mathbb{R}^d)$ and $g \in \mathcal{F}^0(\mathbb{R}^d)$. Taking any step size $0 < s \leq 1/L$, the iterative sequence $\{x_k\}_{k=0}^{\infty}$ generated by ISTA satisfies
    \begin{equation}
      \label{eqn: ista-s}
      \Phi(x_k) - \Phi(x^\star) \leq \frac{\|x_0 - x^\star\|^2}{2sk}, \quad \min_{0\leq i \leq k} \| G_s(x_i) \|^2 \leq \frac{2\|x_0 - x^\star\|^2}{3s^2 k(k+1)};   
    \end{equation}
    Furthermore, the iterative sequence $\{x_k\}_{k=0}^{\infty}$ also satisfies
    \begin{equation}
      \label{eqn: ista-s1}
      \lim_{k\rightarrow \infty} \left[k(k+1)\min_{0\leq i \leq k} \| G_s(x_i) \|^2\right] = 0.
    \end{equation}
  \end{theorem}
\end{frame}

\begin{frame}
  \begin{theorem}[Convergence rate for accelerated case]
    \label{thm: fista}
    Let $\Phi = f + g$ be a composite function with $f \in \mathcal{F}_{L}^1(\mathbb{R}^d)$ and $g \in \mathcal{F}^0(\mathbb{R}^d)$. Taking any step size $0 < s \leq 1/L$, the iterative sequence $\{x_k\}_{k=0}^{\infty}$ generated by FISTA satisfies
    \begin{equation}
      \label{eqn: fista-obj}
      \Phi(x_k) - \Phi(x^\star) \leq \frac{r^2\|y_0 - x^\star\|^2}{2s(k+1)(k+r+1)};   
    \end{equation}
    if the step size satisfies $0< s< 1/L$, the iterative sequence $\{y_k\}_{k=0}^{\infty}$ satisfies
    \begin{equation}
      \label{eqn: fista-grad}
      \left\{\begin{aligned}
      & \min_{0\leq i \leq k} \| G_s(y_i) \|^2 \leq \frac{6r^2\|y_0 - x^\star\|^2}{s^2(1-sL) (k+1)\left(2k^2 + (6r+1)k + 6r^2 \right)},\\
      & \lim_{k\rightarrow \infty} \left(k^3\min_{0\leq i \leq k} \| G_s(y_i) \|^2\right) = 0.
      \end{aligned} \right.
    \end{equation}
    Furthermore, if $r>2$, the iterative sequence $\{x_k\}_{k=0}^{\infty}$ also satisfies
    \begin{equation}
      \label{eqn: fista-obj-fast}
      \lim_{k\rightarrow \infty} \left[k^2 \left(\min_{0\leq i \leq k} \Phi(x_k) - \Phi(x^\star)\right)\right] = 0.
    \end{equation}
  \end{theorem}
\end{frame}


\section{Gradient norm minimization for convex case}
\begin{frame}{Explicity-velocity phase-spase representation}
  We concentrate on accelerated proximal gradient descent method
  \begin{equation*}
    \left\{\begin{aligned}
      & x_{k} = y_{k-1} - sG_s(y_{k-1}),              \\
      & y_{k} = x_{k}  + \frac{k-1}{k+r}(x_{k} - x_{k-1}).
      \end{aligned}\right.
  \end{equation*}
  \pause
  With the velocity iterates
  \begin{equation*}
    v_{k-1}= (y_k - y_{k-1})/\sqrt{s},
  \end{equation*}
  we obtain the gradient-correction phase-space representation
  \begin{equation*}
    \label{phs}
    \left\{\begin{aligned}
      & y_k - y_{k-1} = \sqrt{s}v_{k-1},\\
      & v_k - v_{k-1} = -\frac{r+1}{k}v_k - \left(1 + \frac{r+1}{k}\right)\sqrt{s}G_s(y_k) - \sqrt{s}(G_s(y_k) - G_s(y_{k-1})).
      \end{aligned}\right.
  \end{equation*}
\end{frame}

\begin{frame}{Lyapunov function}
  We construct the discrete Lyapunov function 
  \begin{equation*}
     \mathcal{E}(k) = \underbrace{sk(k+r)(\Phi(x_k) - \Phi(x^\star))}_{\textbf{potential}} + \underbrace{\frac{1}{2}\|\sqrt{s}kv_{k-1} + r(y_k - y^\star) + skG_s(y_{k-1})\|^2}_{\textbf{mixed\ energy}}.
  \end{equation*} 
  \pause
  The continuous Lyapunov function aforementioned
  \begin{equation*}
    \mathcal{E}(t) = t\left(t+\frac{\sqrt{s}}{2}\right)(f(X) - f(x^\star)) + \frac{1}{2}\left\|t\dot{X} + 2(X - x^\star) + t\sqrt{s}\nabla f(X)\right\|^2.
  \end{equation*}
  \pause
  The proof then is calculating the difference $\mathcal{E}(k+1) - \mathcal{E}(k)$.
\end{frame}

\begin{frame}{Convex case}
  First we have
  \begin{align*}
    \mathcal{E}&(k+1) - \mathcal{E}(k) \\
    & =   \underbrace{sk(k+r)\left( \Phi(x_{k+1}) - \Phi(x_{k}) \right) }_{\mathbf{I}_1}+ \underbrace{s(2k+r+1)\left( \Phi(x_{k+1}) - \Phi(x^\star) \right)}_{\mathbf{I}_2}  \\
    & \mathrel{\phantom{=}} - \underbrace{sk(k+r)\left\langle G_s(y_{k}), y_{k} - y_{k-1} \right\rangle }_{\mathbf{II}_1}- \underbrace{sr(k+r)\left\langle G_s(y_{k}),  y_{k} - x^\star\right\rangle}_{\mathbf{II}_2}  \\
    & \mathrel{\phantom{=}} + \frac{s^2(k+r)^2}{2}\big\| G_s(y_{k}) \big\|^2 - \underbrace{s^{2}k(k+r)\langle G_s(y_{k}), G_s(y_{k-1}) \rangle}_{\mathbf{II}_3}
  \end{align*}
  \pause
  By inequality (\ref{ine}), there are 
  \begin{align}
    &\Phi(x_{k+1}) - \Phi(x_k) \le \langle G_s(y_k), y_k - x_k\rangle - \left(s - \frac{s^2 L}{2}\right)\|G_s(y_k)\|^2, \label{eqn: key-1-inq} \\
    &\Phi(x_{k+1}) - \Phi(x^\star) \le \langle G_s(y_k), y_k - x^\star\rangle - \left(s - \frac{s^2 L}{2}\right)\|G_s(y_k)\|^2. \label{eqn: key-2-inq}
\end{align}
\end{frame}

\begin{frame}{Convex case}
  After some cancellation, finally it yields
  \begin{align}
    \mathcal{E}(k+1) - \mathcal{E}(k) \leq & - \frac{s^2(k+r)^2}{2}\left(1- sL\right)\|G_s(y_k)\|^2 \nonumber \\
    & - s \left[ (r-2)k + r^2 - r - 1  \right]\left( \Phi(x_{k+1}) - \Phi(x^\star) \right). 
  \label{eqn: final-iterative}   
  \end{align}
  Thus $\mathcal{E}(k)$ is monotonically decreasing. Our result follows after summing up all the inequalities from $0$ to $k$ and $\lfloor k/2 \rfloor$ to $k$.
\end{frame}


\section{Linear convergence rate for strongly convex objectives}
\begin{frame}{Closer investigation}
  Let us consider the linear inverse problem with $l_1$ regularization 
  \begin{equation}\label{scpro}
    \min_x \|Ax - b\|^2 + \lambda\|x\|_1.
  \end{equation}
  The matrix $A$ is usually ill-conditioned but inversable. Thus (\ref{scpro}) is a strongly convex optimization problem.
\end{frame}

\begin{frame}{Proximal sc-NAG}
	Reconsider
  \begin{equation*}
    \min_x \Phi(x) := f(x) + g(x).
  \end{equation*}
  If $f(x)$ is in addition $\mu$-strongly convex, i.e., 
  \begin{equation}\label{scine}
    f(x) \ge f(y) + \left\langle \nabla f(y), x - y\right\rangle + \frac{\mu}{2}\left\| x - y\right\|^2.
  \end{equation}
  In this case, we consider the proximal sc-NAG scheme 
  \begin{equation}\label{scNAG}
    \left\{\begin{aligned}
      & x_{k} =y_{k-1} - sG_s(y_{k-1}),              \\
      & y_{k} = x_{k}  + \frac{x_{k} - x_{k-1}}{1+2\sqrt{\mu s}}.
    \end{aligned}\right.
  \end{equation}
\end{frame}

\begin{frame}{Numerical experiment}
  We experiment an artificial example of problem 
  \begin{equation*}
    \min_x \|Ax - b\|^2 + \lambda\|x\|_1,
  \end{equation*}
  where we set $A$ as a $500\times 500$ tridiagonal matrix and $b$ as a $500$-dimensional vector,
  \begin{equation*}
    A = \begin{pmatrix}
      2 & 1 &   &   &  \\
      1 & 2 & 1 &   &  \\
        & \ddots & \ddots & \ddots &  \\
        &   & 1 & 2 & 1  \\
        &   &   & 1 & 2 
    \end{pmatrix}\quad
    b = \begin{pmatrix}
      1 \\
      \vdots \\
      1 
    \end{pmatrix}.
  \end{equation*}
  In this case the optimization parameters can be computed as 
  $\mu = \min_{1\leq i \leq d}\lambda_i(A^{T}A) = 1.5461 \times 10^{-9}$ and $L=\max_{1\leq i \leq d} \lambda_i(A^{T}A) = 15.9997$, respectively.
\end{frame}

\begin{frame}{Numerical experiment}
  \begin{figure}[htpb!]
    \centering
    \includegraphics[scale=0.2]{fig/ista_fista.eps}
    \caption{The convergence behavior of the squared proximal subgradient norm by implementing ISTA and FISTA in the least square model with $\ell_1$-regularization.}
    \label{fig: fista}
    \end{figure}
\end{frame}

\begin{frame}{Lyapunov function}
  First consider the implicit-velocity phase-space representation of proximal sc-NAG
  \begin{equation}
    \label{eqn: iv-phase}
    \left\{
    \begin{aligned}
      & x_{k+1} - x_{k} = \sqrt{s}v_{k+1},  \\
      & v_{k+1} - v_{k} = - \frac{2\sqrt{\mu s}v_{k}}{1+2\sqrt{\mu s}} - \sqrt{s}G_s(y_k),
    \end{aligned}
    \right.
    \end{equation}
    where 
    \begin{equation*}
      y_k = x_{k} + \frac{\sqrt{s}v_{k}}{1+2\sqrt{\mu s}}.
    \end{equation*}
    Then construct the discrete Lyapunov function as 
    \begin{equation*}
      \mathcal{E}(k) = \underbrace{\Phi(x_{k}) - \Phi(x^{\star})}_{\mathcal{E}_{\textbf{pot}}(k)} + \underbrace{\frac{\|v_{k}\|^2}{4(1+2\sqrt{\mu s})^2}}_{\mathcal{E}_{\textbf{kin}}(k)} +  \underbrace{\frac{\left\|v_{k} +2\sqrt{\mu}(x_{k} - x^\star)\right\|^2 }{4}}_{\mathcal{E}_{\textbf{mix}}(k)}.
      \end{equation*}
\end{frame}

\begin{frame}{Proof of linear convergence}
  We are able to show that 
  \begin{align*}
    \mathcal{E}(k+1) - \mathcal{E}(k) \leq & -\sqrt{\mu s} \min\left\{1, \frac{1+\sqrt{\mu s}}{\frac{3}{4} + \sqrt{\mu s} + \mu s}, \frac14 \right\}\mathcal{E}(k+1) \\
    & -\frac{s (1-sL) }{2} \|G_s(y_k)\|^2.
  \end{align*}
  With initial value $x_0\in \mathbb{R}^n$ and $v_0 = 0$, we get the conclusion.
\end{frame}

\begin{frame}{Linear convergence}
  \begin{theorem}
    Let $\Phi = f+ g$ be a composite function with $f \in \mathcal{S}_{\mu,L}^{1}(\mathbb{R}^d)$ and $g \in \mathcal{F}^0(\mathbb{R}^d)$. Taking any step size $0 < s \leq 1/L$, the iterative sequence $\{x_k\}_{k=0}^{\infty}$ generated by FISTA satisfies
    \begin{equation}
    \label{eqn: fista_rate_obj_1}
    \Phi(x_{k}) - \Phi(x^\star) \leq \frac{\Phi(x_0) - \Phi(x^\star) + \mu\|x_0 - x^\star\|^2}{\left(1 + \frac{\sqrt{\mu s}}{4}\right)^k};
    \end{equation}
    if the step size satisfies $0 < s < 1/L$, the iterative sequence $\{y_k\}_{k=0}^{\infty}$ satisfies
    \begin{equation}
    \label{eqn: fista_rate_gn_1}
    \|G_s(y_k)\|^2 \leq \frac{2L\left(\Phi(x_0) - \Phi(x^\star) + \mu\|x_0 - x^\star\|^2\right)}{(1-sL)\left(1 + \frac{\sqrt{\mu s}}{4}\right)^k}.
    \end{equation}
    \end{theorem}
\end{frame}


\section{Research plan on game theory}  
\begin{frame}{Overview}
	Game theory is a large area of subject. It research on the descision behavior of multiple players (or agents) against or cooperate with each other. This can be viewed as a generalization of optimization problem, where optimization has only one player while a game has more than two. \par 
  
  One discipline is \emph{differential games} which considers the game of several agents under the contraints of specified continuous dynamical systems. Examples are like isotropic rocket game, optimal program of steel production, etc..
\end{frame}

\begin{frame}{Differential games}
  Homicidal chauffeur game 
  \begin{figure}[htpb!]
    \centering
    \includegraphics[scale=0.36]{fig/homicidal chauffeur game.png}
    \caption{The swerve phenomenon}
    \end{figure}
\end{frame}
\begin{frame}{Differential games}
  The optimal program of steel production 
  \begin{figure}[htpb!]
    \centering
    \includegraphics[scale=0.36]{fig/steel production.png}
    \caption{The division of state spase}
    \end{figure}
\end{frame}

\begin{frame}{Procedure of solving differential games}
  First, call the two players $P$ and $E$. Specify the state variable $\x$, state space $\E$, and the control variables $\phi = (\phi_1,\ldots,\phi_\lambda)$ and $\varphi = (\varphi_1,\ldots,\phi_\mu)$ and their constraints. Suppose the dimension of $\E$ is $n$. Also specify the terminal surface $\C$. \par 

  Second, write down the kinematic equations (KE) of the problem as 
  \begin{equation}
    \dot{\x} = f(\x, \phi, \varphi),
  \end{equation}
  where $\x\in \E$, $f = (f_1,\ldots,f_n)^T$, and $\dot{\x} = \frac{\dd \x}{\dd t}$ is the time derivative. Also the payoff function is claimed as 
  \begin{equation}
    L(\x, \phi, \varphi) = \int G(\x, \phi, \varphi) \dd t + H(\x_T).
  \end{equation}
  Here $\x_T$ is the terminal position and $H(\x_T)$ is called the terminal payoff.
\end{frame}

\begin{frame}{Procedure of solving differential games}
  The value of game is defined as 
  \begin{equation}
    V(\x) = \min_{\phi} \max_{\varphi} L(\x,\phi,\varphi).
  \end{equation}
  \par 
  Third, deduce the first main equation (ME$_1$)
  \begin{equation}
    \min_{\phi} \max_{\varphi} V_i f_i(\x,\phi,\varphi) + G(\x,\phi,\varphi) = 0,
  \end{equation}
  and solve the minimax by
  \begin{equation}
    \bar{\phi}(\x,V_x)\quad \bar{\varphi}(\x,V_x),
  \end{equation}
  then we get the second mean equation (ME$_2$) 
  \begin{equation}
    V_{x_i}f_i(\x,\bar{\phi},\bar{\varphi}) + G(\x,\bar{\phi},\bar{\varphi}) = 0.
  \end{equation}
\end{frame}

\begin{frame}{Procedure of solving differential games}
  Fourth, if $\phi$ and $\varphi$ are subject to constant bounded, write out the Retrogressive path equations (RPE) 
  \begin{align}
    & \mathring{\x_k} = -f_k(\x,\bar{\phi},\bar{\varphi})\\
    & \mathring{V_k} = \sum_i V_i f_{ik}(\x,\bar{\phi},\bar{\varphi}) + G_k(\x,\bar{\phi},\bar{\varphi}).
  \end{align}
  Or else we can obtain a more general RPE by differentiating ME$_2$ with respect to time variable $t$. Note that in this case, if one goes through the procedure of deriving the RPE again, it can be found that the right hand side of RPE for $\mathring{V_k}$ is just the partial derivative of $x_k$ of ME$_2$, viewing $V_x$ as constant and other $f$, $G$, $\bar{\phi}$, and $\bar{\varphi}$ as functions of $\x$.
\end{frame}

\begin{frame}{Procedure of solving differential games}
  Fifth, calculate the initial conditions. let $\nu = (\nu_1,\ldots,\nu_n)$ be the vector normal to $\C$ from point $\x$ on $\C$ and extending into $\E$. Solve the useable part of $\C$ by 
  \begin{equation}
    \min_{\phi} \max_{\varphi} \sum_i \nu_i f_i(\x,\phi,\varphi) < 0.
  \end{equation}
  Parameterize the terminal surface as 
  \begin{equation}
    x_i = h_i(s_i,\ldots,s_{n-1}),
  \end{equation}
  then we have the initial condition for $x_i$ given $s_i$. For $V_i$, use the fact $V = H = H(s_1,\ldots,s_{n-1})$ on $\C$, we have 
  \begin{equation}
    \frac{\partial H}{\partial s_k} = \sum_i V_i \frac{\partial h_i}{\partial s_k}, \quad k = 1,\ldots,n-1.
  \end{equation}
  Combined with ME$_2$ on $\C$, we thus solve $V_i$. \par 
  Finally, integrating the RPE gives the rough solution.
\end{frame}

\begin{frame}{Equilibrium problem}
  The Nash equilibrium is a point $x^\star = (x_1^\star,\ldots,x_n^\star)$ that satisfies 
  \begin{equation*}
    \min_{x_i\in \mathcal{C}_i} f_i(x_1^\star,\ldots, x_{i-1}^\star, x_i, x_{i+1}^\star,\ldots, x_n^\star)
  \end{equation*}
  For two-player zero-sum games, 
  \begin{equation*}
    \max_x \min_y L(x, y),
  \end{equation*}
  there are algorithms like gradient ascent descent method (GAD)
  \begin{equation*}
    \left\{\begin{aligned}
      & x_{k+1} = x_k + \eta \nabla_x L(x_k, y_k),\\
      & y_{k+1} = y_k - \eta \nabla_y L(x_k, y_k).
    \end{aligned}\right.
  \end{equation*}
\end{frame}

\section*{Acknowledgement}
\begin{frame}
\textcolor{myNewColorA}{\Huge{\centerline{Thank you!}}}
\end{frame}

\section*{Reference}
\begin{frame}
{\small
\bibliographystyle{abbrvnat}
\bibliography{reference}
}
\end{frame}

\end{document}



