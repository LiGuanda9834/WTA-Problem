\documentclass[10pt]{beamer}
%\usepackage[utf8]{inputenc}
%\usepackage{xeCJK}
\usepackage{graphicx}
\usepackage{subcaption}
\usepackage {mathtools}
%\usepackage{utopia} %font utopia imported
\usetheme{CambridgeUS}
\usecolortheme{dolphin}

% set colors
\definecolor{myNewColorA}{RGB}{25,25,112}
\definecolor{myNewColorB}{RGB}{25,25,112}
\definecolor{myNewColorC}{RGB}{25,25,112}
\setbeamercolor*{palette primary}{bg=myNewColorC}
\setbeamercolor*{palette secondary}{bg=myNewColorB, fg = white}
\setbeamercolor*{palette tertiary}{bg=myNewColorA, fg = white}
\setbeamercolor*{titlelike}{fg=myNewColorA}
\setbeamercolor*{title}{bg=myNewColorA, fg = white}
\setbeamercolor*{item}{fg=myNewColorA}
\setbeamercolor*{caption name}{fg=myNewColorA}
\usefonttheme{professionalfonts}
\usepackage{natbib}
\usepackage{hyperref}
%------------------------------------------------------------
%\titlegraphic{\includegraphics[height=1.5cm]{logo.png}} 

\setbeamerfont{title}{size=\large}
\setbeamerfont{subtitle}{size=\small}
\setbeamerfont{author}{size=\small}
\setbeamerfont{date}{size=\small}
\setbeamerfont{institute}{size=\small}



%设置页码脚标
%\setbeamertemplate{footline}[frame number]
\defbeamertemplate{footline}{NGEGFootlineTemplate}{%
	\leavevmode% 离开vmode,也就是离开竖直模式,进入水平模式
	\hbox{%
		\begin{beamercolorbox}[wd=.2\paperwidth,ht=2.25ex,dp=1ex,center]{author in head/foot}%
			\ifnum \the\value{page}>1 \usebeamerfont{author in head/foot}\insertshortauthor\fi
		\end{beamercolorbox}%
		\begin{beamercolorbox}[wd=.6\paperwidth,ht=2.25ex,dp=1ex,center]{title in head/foot}%
			\ifnum \the\value{page}>1 \usebeamerfont{title in head/foot}\insertshorttitle\fi
		\end{beamercolorbox}%
		\begin{beamercolorbox}[wd=0.2\paperwidth,ht=2.25ex,dp=1ex,center]{author in head/foot}%
			\ifnum \the\value{page}>1 \insertframenumber{} / \inserttotalframenumber\fi
	\end{beamercolorbox}}%
%	\vskip0pt%
}
\setbeamertemplate{footline}[NGEGFootlineTemplate]



\newcommand{\mb}[1]{\mathbf{#1}}
\newcommand{\mc}[1]{\mathcal{#1}}
\newcommand{\E}{\mathcal{E}}
\newcommand{\C}{\mathcal{C}}
\newcommand{\x}{\mathbf{x}}
\newcommand{\dd}{\mathrm{d}}

\newcommand{\sumFromTo}[3]{\ensuremath{\sum_{#1}^{#2} #3}}

\newcommand{\optimalProblem}[3]
{\begin{align*}
    #1 \quad &#2 \\
    \mathrm{s. t.}\quad&#3
\end{align*}}




\title[Weapon Target Assignment problem]{Literature Review of Weapon Target Assignment problem and a Branch-Bound-Cut algorithm on it}%主标题
%\subtitle{ }%%副标题
\author[Guanda Li]{Guanda Li
\and joint work with Liang Chen and Yu-hong Dai}%%作者

\institute[LSEC]{Institute of Computational Mathematics and Scientific/Engineering Computing,\\
Academy of Mathematics and Systems Science,\\
Chinese Academy of Sciences}
\date[\textcolor{white} ]
{December 20, 2022}

%------------------------------------------------------------
%This block of commands puts the table of contents at the 
%beginning of each section and highlights the current section:
%\AtBeginSection[]
%{
%  \begin{frame}
%    \frametitle{Contents}
%    \tableofcontents[currentsection]
%  \end{frame}
%}
\AtBeginSection[]{
  \begin{frame}
  \vfill
  \centering
  \begin{beamercolorbox}[sep=8pt,center,shadow=true,rounded=true]{title}
    \usebeamerfont{title}\insertsectionhead\par%
  \end{beamercolorbox}
  \vfill
  \end{frame}
}
%------------------------------------------------------------

\begin{document}

%The next statement creates the title page.
\frame{\titlepage}
\begin{frame}
\frametitle{Contents}
\tableofcontents
\end{frame}


\section{Background}
\begin{frame}
    \frametitle{derivation}
    \textbf{背景介绍}
        \begin{itemize}
            \item \textbf{Weapon Target Assignment problem}是一个军事领域中的问题
            \item 其基本问题是考虑使用m个武器攻击n个目标,以最小化所有目标的加权存活概率.{\tiny(或等价地,对目标造成的加权伤害最大化)}
            \item \textbf{该问题被证明是NP-C的},并且通常被表述为非线性模型。
            \item 在80个武器和80个目标的问题计算需要16.2个小时。
            \item 将问题\textbf{线性化}后可以采用\textbf{列生成}求解。
        \end{itemize}
\end{frame}

\begin{frame}
    \frametitle{history}
\end{frame}

\begin{frame}
    \frametitle{basic formulation}
    \begin{columns}
        \begin{column}{.70\linewidth}
            \footnotesize
            \begin{itemize}
                \item $I = \left\{1,\cdots,m\right\} $, 武器集合.
                \item $J = \left\{1,\cdots,n\right\} $, 目标集合.
                \item $p_{ij}\in [0,1]$, $i$击落$j$的概率.
            \end{itemize}
        \end{column}
    \hspace{-1cm}
        \begin{column}{.35\linewidth}
            \footnotesize
            \begin{itemize}
                \item $a_j$, 目标$j$的权重.
                \item $x_{ij}$,武器$i$是否攻击$j$.
            \end{itemize}
        \end{column}
    \end{columns}
    
    \begin{align}
        \min\quad & \sum_{j=1}^n a_j \left( \prod_{i=1}^m (1 -  p_{ij})^{x_{ij}} \right) \\ 
        \mathrm{s. t.}\quad &\sum_{j=1}^n x_{ij} \leq 1\quad \forall ~i \in I,\\
        & x_{ij} \in \left\{ 0,1 \right\} \quad \forall~ j\in J , ~ i \in I.
    \end{align}
\end{frame}

\section{Literature Review}
\subsection{Formulations}
\begin{frame}
    \frametitle{static WTA model 1}
\end{frame}

\subsection{Algorithms}
\begin{frame}
    \frametitle{dynamic WTA model 1}
\end{frame}

\subsection{the Latest Development}
\begin{frame}
    \frametitle{Semsor WTA model}
\end{frame}

\section{useful algorithms}
\begin{frame}
    \frametitle{basic formulation}
    \begin{columns}
        \begin{column}{.70\linewidth}
            \footnotesize
            \begin{itemize}
                \item $I = \left\{1,\cdots,m\right\} $, 武器集合.
                \item $J = \left\{1,\cdots,n\right\} $, 目标集合.
                \item $p_{ij}\in [0,1]$, $i$击落$j$的概率.
            \end{itemize}
        \end{column}
    \hspace{-1cm}
        \begin{column}{.35\linewidth}
            \footnotesize
            \begin{itemize}
                \item $a_j$, 目标$j$的权重.
                \item $x_{ij}$,武器$i$是否攻击$j$.
            \end{itemize}
        \end{column}
    \end{columns}
    
    \begin{align}
        \min\quad & \sum_{j=1}^n a_j \left( \prod_{i=1}^m (1 -  p_{ij})^{x_{ij}} \right) \\ 
        \mathrm{s. t.}\quad &\sum_{j=1}^n x_{ij} \leq 1\quad \forall ~i \in I,\\
        & x_{ij} \in \left\{ 0,1 \right\} \quad \forall~ j\in J , ~ i \in I.
    \end{align}
\end{frame}

\begin{frame}
    \frametitle{formulation}
    \begin{equation*}
        \prod_{i=1}^{m}{(1-p_{ij}x_{ij})} \iff \prod_{i=1}^{m}{(1-p_{ij})^{x_{ij}}}, x \in \left\{ 0,1 \right\}
    \end{equation*}
    \begin{itemize}
        \item 当两个问题限定$x$取值为$0,1$时,取值相等
        \item 后一个形式进行松弛后问题是凸的(可以通过简单的求二阶导得到),因此对该问题可以采用外逼近算法。
    \end{itemize}
\end{frame}

\begin{frame}
    \frametitle{目标函数为多个凸函数的加权和}
    考虑目标函数中的一项:
    \begin{equation*}
        f_j(x) =  \prod_{i=1}^m (1 -  p_{ij})^{x_{ij}}\quad \forall~  j \in J
    \end{equation*}
    考虑该问题的Hessian矩阵$H$:
    \begin{equation*}
        \frac{\partial f_j(x)}{\partial x_{aj} \partial x_{bj}} = \ln(1 - p_{aj}) \ln(1 - p_{bj}) f_j(x)
    \end{equation*}
    {
    \scriptsize
    \begin{equation*}
        H = f(x)\begin{bmatrix}
            \ln(1 - p_{1j}) \ln(1 - p_{1j}) & \ln(1 - p_{1j}) \ln(1 - p_{2j}) &\cdots & \ln(1 - p_{1j}) \ln(1 - p_{mj}) \\
            \ln(1 - p_{2j}) \ln(1 - p_{1j}) & \ln(1 - p_{2j}) \ln(1 - p_{2j}) &\cdots & \ln(1 - p_{2j}) \ln(1 - p_{mj}) \\
            \vdots & \vdots &\ddots &\vdots \\
            \ln(1 - p_{mj}) \ln(1 - p_{1j}) & \ln(1 - p_{mj}) \ln(1 - p_{2j}) &\cdots & \ln(1 - p_{mj}) \ln(1 - p_{mj})


        \end{bmatrix}
    \end{equation*}
    }
\end{frame}

\begin{frame}
    \frametitle{目标函数为多个凸函数的加权和}
    若记
    \begin{equation*}
        l = \begin{bmatrix}
            \ln(1 - p_{1j}) & \ln(1 - p_{2j}) & \cdots & \ln(1 - p_{mj})
        \end{bmatrix}
    \end{equation*}
    可得到Hassien矩阵的形式为:
    \begin{equation*}
        H = f(x)l\cdot l^T
    \end{equation*}
    可以看出该矩阵是个秩一矩阵,即各列之间是线性相关的,从而其行列式为0,从而得到原函数是凸的。
\end{frame}


\subsection{Columnn Generation Algorithm}
\begin{frame}
    \frametitle{Basic Idea}
    
\end{frame}

\begin{frame}
    \frametitle{How to use column generation in WTA Problem}
    \begin{itemize}
		\item \textbf{基本思路}:通过将所有目标场景$S$列出来的方式,将问题转化为一个线性规划问题。
		\item \textbf{转化方式}:假设一共有$m$个武器,则对于任意一个目标$j$,每个武器可以选择打击$j$或不打击$j$,因此共有$2^m$个不同的打击方案。
		\item \textbf{一个例子}:一共有8个武器,则使用第1,3,6号武器的攻击方案即记为$s_{[1,0,1,0,0,1,0,0]}$,$|S| = 2^8 = 256$
		\item $n_{si}$:0-1变量,表示在第$s$个场景下,是否启用武器$i$. 在上面的例子中,$n_{s1} = 1,\ n_{s2} = 0$
		\item $q_{js} = a_j \prod_{i = 1}^m (1 - n_{si}\cdot p_{ij})$:使用第方案$s$打击目标$j$的概率并加权,比如$q_{3s}$就是采用1,3,6号武器打击$3$号目标的击毁概率乘以目标$j$的权重。
	\end{itemize}
\end{frame}

\begin{frame}
    \frametitle{transformed formulation}
    \optimalProblem{\min}{\sumFromTo{j = 1}{n}{\sumFromTo{s = 0}{2^m -1}{q_{js}y_{js}}}}{\sumFromTo{j = 1}{n}{\sumFromTo{s = 0}{2^m -1}{n_{si}y_{js}}}\leq 1 \quad &\forall ~ i \in I\\& \sumFromTo{s = 0}{2^m - 1}{y_{js}} = 1 \quad &\forall ~ j \in J\\& y_{js} \in \left\{ 0,1 \right\} &\forall ~ j\in J,\ s\in S}
    \begin{columns}
        \begin{column}{.70\linewidth}
            \footnotesize
            \begin{itemize}
                \item $I = \left\{1,\cdots,m\right\} $, 武器集合.
                \item $J = \left\{1,\cdots,n\right\} $, 目标集合.
                \item $S = \left\{1,\cdots,2^m\right\}$, 场景集合.
            \end{itemize}
        \end{column}
    \hspace{-3cm}
        \begin{column}{.70\linewidth}
            \footnotesize
            \begin{itemize}
                \item $n_{si}$:场景$s$是否使用武器$i$
                \item $y_{js}$:0-1变量,是否对目标$j$应用$s$
                \item $q_{js}$:赋权后的场景$s$对$j$的摧毁概率
            \end{itemize}
        \end{column}
    \end{columns}
\end{frame}

\begin{frame}
    \frametitle{transformed formulation continuous}
    \optimalProblem{\min}{\sumFromTo{j = 1}{n}{\sumFromTo{s = 0}{2^m -1}{q_{js}y_{js}}}}{\sumFromTo{j = 1}{n}{\sumFromTo{s = 0}{2^m -1}{n_{si}y_{js}}}\leq 1 \quad &\forall ~ i \in I\\& \sumFromTo{s = 0}{2^m - 1}{y_{js}} = 1 \quad &\forall ~ j \in J\\& y_{js} \in \left\{ 0,1 \right\} & \forall ~ j\in J,\ s\in S}
    \begin{itemize}
        \item \textbf{目标函数}:最小化赋权后的消灭概率
        \item \textbf{第一个约束}:每个武器最多只能攻击一个目标
        \item \textbf{第二个约束}:每个目标恰好安排一个场景
    \end{itemize}
\end{frame}

\begin{frame}
    \frametitle{LP Relaxition}
    \optimalProblem{\min}{\sumFromTo{j = 1}{n}{\sumFromTo{s = 0}{2^m -1}{q_{js}y_{js}}}}{\sumFromTo{j = 1}{n}{\sumFromTo{s = 0}{2^m -1}{n_{si}y_{js}}}\leq 1 \quad &\forall ~ i \in I\\& \sumFromTo{s = 0}{2^m - 1}{y_{js}} = 1 \quad &\forall ~  j \in J\\& y_{js} \geq 0 &\forall ~ j\in J,\ s\in S}
    \begin{itemize}
        \item 将问题进行线性松弛,第三个约束可以直接放成$y_{js} \geq 0$
        \item 此时线性规划的特点为包含了$n \times 2^m$列
        \item 列生成思路:仅取其中的部分列,并转化为对偶问题。
    \end{itemize}
\end{frame}

\begin{frame}
    \frametitle{Dual Problem}
    \optimalProblem{\max}{\sumFromTo{i=1}{m}{u_i} + \sumFromTo{j=1}{n}{v_j}}{\sumFromTo{i=1}{m}{x_{is}u_i}+ v_j \leq q_{js}\quad (P) \\ &u_i \leq 0,\quad v_j\ \mathrm{free}}
    \begin{itemize}
        \item 选择原问题中的部分列相当于对偶问题选择了部分行。
        \item 求解得到$u^*, v^*$并带入原对偶,若都成立则一定为最优解。
        \item 否则希望选择一个违背最严重的,即$\sumFromTo{i=1}{m}{x_{is}u_i}+ v_j > q_{js}$中最大的。
    \end{itemize}
\end{frame}

\begin{frame}
    \frametitle{Restricted Dual Problem}
    \optimalProblem{\max}{\sumFromTo{i=1}{m}{u_i} + \sumFromTo{j=1}{n}{v_j}}{\sumFromTo{i=1}{m}{x_{is}u_i}+ v_j \leq q_{js}\quad (P) \\ &u_i \leq 0,\quad v_j\ \mathrm{free}}
\end{frame}

\begin{frame}
    \frametitle{subproblem}
    \optimalProblem{\min}{a_j \prod_{i=1}^{m}{(1-p_{ij}x_{is})} -\sumFromTo{i=1}{m}{x_{is}}u_i - v_j}{j \in J,\quad s \in S}
    \begin{itemize}
        \item (接上文)希望选择一个违背最严重的,即$\sumFromTo{i=1}{m}{x_{is}u_i}+ v_j > q_{js}$中最大的。
        \item 将$q_{js}$的定义带入,即可得到上述子问题。
        \item 可以将问题分离,之后对每个给定的目标$j$求解,其直观为使用每个武器攻击需要一个花费,希望平衡开启武器的花费和消灭目标的概率。
    \end{itemize}
\end{frame}




\subsection{Outer Approximation Algorithm}
\begin{frame}
    \frametitle{transformed model}
    原模型的目标函数为:
    \begin{equation*}
        \sum_{j=1}^n a_j \left( \prod_{i=1}^m (1 -  p_{ij})^{x_{ij}} \right)
    \end{equation*}
    {\footnotesize
    希望通过引入辅助变量的形式,将非线性项从目标函数转化到约束中。
    
    若记$\eta_j$ 为代表 $\prod_{i=1}^m (1 -  p_{ij})^{x_{ij}}$引入的变量,原问题可转化为:
    \begin{align}
        \min\quad & \sum_{j=1}^n a_j \eta_j \\ 
        \mathrm{s. t.}\quad & \eta_j \geq \prod_{i=1}^m (1 -  p_{ij})^{x_{ij}}, \quad \forall ~ j \in J \\ 
        &\sum_{j=1}^n x_{ij} \leq 1\quad \forall ~ i \in I,\\
        & x_{ij} \in \left\{ 0,1 \right\} \quad \forall ~ j\in J\quad i \in I.
    \end{align}
    }
\end{frame}

\begin{frame}
    \frametitle{Basic idea of outer approximation}
    若$f(x)$为凸函数,则对于可行域中的任意一个点$x^*$,有
    \begin{equation*}
        f(x) \geq f(x^*) + \nabla f(x^*)(x - x^*)
    \end{equation*}
    因此若原问题的约束为$\eta \geq f(x)$,则此时
    \begin{equation*}
        \eta \geq f(x) \geq f(x^*) + \nabla f(x^*)(x - x^*)
    \end{equation*}
    一定成立。

    即对于任意一个给定的点,都可以引入一个线性约束,保证可行域满足该约束。

    外逼近方法即希望利用整数特性,通过引入多个线性约束来替代某个非线性约束,但保证转化前后问题是等价的。
\end{frame}

\begin{frame}
    \frametitle{constraint formulation of outer approximation}
    仍记$f_j(x) =  \prod_{i=1}^m (1 -  p_{ij})^{x_{ij}}\quad j \in J$,则对于任意给定的点$\bar{x}$,有
    \begin{align*}
        \nabla f(\bar{x})(x - \bar{x}) & = f(\bar{x})\sum_{i = 1}^m \ln(1-p_{ij})(x_{ij} - \bar{x_{ij}})\\
        &= f(\bar{x})\sum_{i = 1}^m \ln(1-p_{ij})x_{ij} - f(\bar{x})\sum_{i = 1}^m \ln(1-p_{ij})\bar{x_{ij}}
    \end{align*}
    其中前一项是变量的线性组合,有一项在$\bar{x}$给定时是常数。依据此外逼近约束的形式,即可得到问题的外逼近模型:
\end{frame}

\begin{frame}
    \frametitle{outer approximation model}
    若记问题中的全部整数可行解为集合$X$,则模型可以描述为
    {\scriptsize
    \begin{align}
        \min\quad & \sum_{j=1}^n a_j \eta_j \\ 
        \mathrm{s. t.}\quad & \eta_j \geq f(\bar{x})\sum_{i = 1}^m \ln(1-p_{ij})(x_{ij} - \bar{x_{ij}}) + f(\bar{x}), \quad \forall ~ j \in J,\ \bar{x} \in X \\ 
        &\sum_{i=1}^m x_{ij} \leq 1\quad \forall ~ j \in J,\\
        & x_{ij} \in \left\{ 0,1 \right\} \quad \forall ~ j\in J\quad i \in I.
    \end{align}
    }
    可以看出,此时约束的个数非常多,对问题的求解造成了一定的困难,因此考虑限制后的模型,即近考虑部分约束。
\end{frame}

\begin{frame}
    \frametitle{Idea of outer approximation method}
    {
    \scriptsize
    \begin{align*}
        \min\quad & \sum_{j=1}^n a_j \eta_j \\ 
        \mathrm{s. t.}\quad & \eta_j \geq f(\bar{x})\sum_{i = 1}^m \ln(1-p_{ij})(x_{ij} - \bar{x_{ij}}) + f(\bar{x}), \quad j \in J,\ \bar{x} \in X \\ 
        &\sum_{i=1}^m x_{ij} \leq 1\quad j \in J,\quad x_{ij} \in \left\{ 0,1 \right\} \quad j\in J\quad i \in I
    \end{align*}
    基本思路:约束(12)太多了,假设仅需要部分约束即可找到最优解,借此降低搜索规模

    \textbf{Idea}
    \begin{enumerate}
        \item 去除全部约束(12),给出当前最优解$x, \eta$
        \item 检验当前最优解是否能满足全部的约束(12),若可以,迭代终止,执行结束
        \item 否则选择违背最严重的约束(12),加入到其中,重新求解并执行步骤2
    \end{enumerate}
    }
\end{frame}

\begin{frame}
    \title{choose violate constraint}
    \begin{align*}
        \min\quad & \sum_{j=1}^n a_j \eta_j \\ 
        \mathrm{s. t.}\quad & \eta_j \geq f(\bar{x})\sum_{i = 1}^m \ln(1-p_{ij})(x_{ij} - \bar{x_{ij}}) + f(\bar{x}), \quad j \in J,\ \bar{x} \in X \\ 
        &\sum_{i=1}^m x_{ij} \leq 1\quad j \in J,\quad x_{ij} \in \left\{ 0,1 \right\} \quad j\in J\quad i \in I
    \end{align*}
    \begin{itemize}
        \item 除非已经找到了最优解,否则限制后的问题的解一定会带来一个违背的约束(放大了可行域,目标函数一定会更小,说明它不在原问题的可行域当中)
        \item 可以直观地理解,违背即指$\eta_j$在可行域之外了,由于本身是求的最小化的问题,因此在找到的点加割即为最违背的割。
        \item 由此可以得到当求解出一个在可行域当中的点时,如果$\eta < f_j(x)$,就说明这个点应该被包含在可行域之外,因此可以在这个点加入一个割。
    \end{itemize}
\end{frame}


\subsection{Branch-bound-cut Algorithm}
\begin{frame}
    \frametitle{Branch-Bound-Cut Algorithm}
    {\scriptsize
    \begin{align}
        \min\quad & \sum_{j=1}^n a_j \eta_j \\ 
        \mathrm{s. t.}\quad & \eta_j \geq \prod_{i=1}^m (1 -  p_{ij})^{x_{ij}}, \quad \forall ~ j \in J \\ 
        &\sum_{j=1}^n x_{ij} \leq 1\quad \forall ~ i \in I,\\
        & x_{ij} \in \left\{ 0,1 \right\} \quad \forall ~ j\in J\quad i \in I.
    \end{align}
    }
    \begin{itemize}
        \item 仍然考虑该模型
    \end{itemize}
\end{frame}


\begin{frame}
    \frametitle{Basic Idea}
    {\scriptsize
    
    \begin{align}
        \min\quad & \sum_{j=1}^n a_j \eta_j \\ 
        \mathrm{s. t.}\quad & \eta_j \geq f(\bar{x})\sum_{i = 1}^m \ln(1-p_{ij})(x_{ij} - \bar{x_{ij}}) + f(\bar{x}), \quad \forall ~ j \in J,\ \bar{x} \in X \\ 
        &\sum_{i=1}^m x_{ij} \leq 1\quad \forall ~ j \in J,\\
        & x_{ij} \in \left\{ 0,1 \right\} \quad \forall ~ j\in J\quad i \in I.
    \end{align}
    }
    \begin{itemize}
        \item 受到上一个方法的启发,问题最优解可以在仅有部分切平面的约束时取到,
        \item 当取到最优解时,全部的约束都应该满足,如果某个约束不被满足,则求解得到的节点应该被某个约束割掉。
        \item 整体采用分支定界的框架,求解去除所有约束的问题。在求解得到某个节点时,对该节点进行判断,若此时不满足上述约束,则通过加割来割掉该节点
    \end{itemize}
\end{frame}
\begin{frame}
    \frametitle{a potential improvement}
    {\scriptsize
    \begin{align}
        \min\quad & \sum_{j=1}^n a_j \eta_j \\ 
        \mathrm{s. t.}\quad & \eta_j \geq f(\bar{x})\sum_{i = 1}^m \ln(1-p_{ij})(x_{ij} - \bar{x_{ij}}) + f(\bar{x}), \quad \forall ~ j \in J,\ \bar{x} \in X \\ 
        &\sum_{i=1}^m x_{ij} \leq 1\quad \forall ~ j \in J,\\
        & x_{ij} \in \left\{ 0,1 \right\} \quad \forall ~ j\in J\quad i \in I.
    \end{align}
    }
    \begin{itemize}
        \item 在外逼近算法中,仅在计算出整数解之后才往里添加割
        \item 在分支定界算法中,每次求解LP松弛问题都会得到一个松弛解,该松弛解也可以给问题带来一个割。
        \item 由于问题是凸的,因此这些分数割都是有效的不等式,即不会把原问题中的可行解割掉
        \item 数值实验主要对比了仅对整数解加割和同时对分数解也加割的效果
    \end{itemize}
\end{frame}


\section{computational result of branch-bound-cut algorithm}
\begin{frame}
    \frametitle{Numerical Experiment Environment}
\end{frame}

\begin{frame}
    \frametitle{only use the feasible solution cut}
\end{frame}

\begin{frame}
    \frametitle{also use the fractional silution cut}
\end{frame}

\begin{frame}
    \frametitle{camparition of two situations}
\end{frame}

\begin{frame}
    \frametitle{camparition with other methods}
\end{frame}





\section{Future Research Plan}

\section{Acknowledgement and Reference}
\begin{frame}
\textcolor{myNewColorA}{\Huge{\centerline{Thank you!}}}
\end{frame}

\begin{frame}
{\small
\bibliographystyle{abbrvnat}
\bibliography{reference}
}
\end{frame}


\end{document}