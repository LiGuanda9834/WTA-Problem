\documentclass[CJK,10pt]{beamer}
%\usepackage[utf8]{inputenc}
\usepackage{xeCJK}
\usepackage{graphicx}
\usepackage{subcaption}
\usepackage {mathtools}
%\usepackage{utopia} %font utopia imported
\usetheme{CambridgeUS}
\usecolortheme{dolphin}

% set colors
\definecolor{myNewColorA}{RGB}{25,25,112}
\definecolor{myNewColorB}{RGB}{25,25,112}
\definecolor{myNewColorC}{RGB}{25,25,112}
\setbeamercolor*{palette primary}{bg=myNewColorC}
\setbeamercolor*{palette secondary}{bg=myNewColorB, fg = white}
\setbeamercolor*{palette tertiary}{bg=myNewColorA, fg = white}
\setbeamercolor*{titlelike}{fg=myNewColorA}
\setbeamercolor*{title}{bg=myNewColorA, fg = white}
\setbeamercolor*{item}{fg=myNewColorA}
\setbeamercolor*{caption name}{fg=myNewColorA}
\usefonttheme{professionalfonts}
\usepackage{natbib}
\usepackage{hyperref}
%------------------------------------------------------------
%\titlegraphic{\includegraphics[height=1.5cm]{logo.png}} 

\setbeamerfont{title}{size=\large}
\setbeamerfont{subtitle}{size=\small}
\setbeamerfont{author}{size=\small}
\setbeamerfont{date}{size=\small}
\setbeamerfont{institute}{size=\small}



%设置页码脚标
%\setbeamertemplate{footline}[frame number]
\defbeamertemplate{footline}{NGEGFootlineTemplate}{%
	\leavevmode% 离开vmode,也就是离开竖直模式,进入水平模式
	\hbox{%
		\begin{beamercolorbox}[wd=.2\paperwidth,ht=2.25ex,dp=1ex,center]{author in head/foot}%
			\ifnum \the\value{page}>1 \usebeamerfont{author in head/foot}\insertshortauthor\fi
		\end{beamercolorbox}%
		\begin{beamercolorbox}[wd=.6\paperwidth,ht=2.25ex,dp=1ex,center]{title in head/foot}%
			\ifnum \the\value{page}>1 \usebeamerfont{title in head/foot}\insertshorttitle\fi
		\end{beamercolorbox}%
		\begin{beamercolorbox}[wd=0.2\paperwidth,ht=2.25ex,dp=1ex,center]{author in head/foot}%
			\ifnum \the\value{page}>1 \insertframenumber{} / \inserttotalframenumber\fi
	\end{beamercolorbox}}%
%	\vskip0pt%
}
\setbeamertemplate{footline}[NGEGFootlineTemplate]



\newcommand{\mb}[1]{\mathbf{#1}}
\newcommand{\mc}[1]{\mathcal{#1}}
\newcommand{\E}{\mathcal{E}}
\newcommand{\C}{\mathcal{C}}
\newcommand{\x}{\mathbf{x}}
\newcommand{\dd}{\mathrm{d}}

\newcommand{\sumFromTo}[3]{\ensuremath{\sum_{#1}^{#2} #3}}

\newcommand{\optimalProblem}[3]
{\begin{align*}
    #1 \quad &#2 \\
    \mathrm{s. t.}\quad&#3
\end{align*}}




\title[Weapon Target Assignment problem]{Literature Review of Weapon Target Assignment problem}%主标题
%\subtitle{ }%%副标题
\author[Guanda Li]{Guanda Li
}%%作者

\institute[LSEC]{Institute of Computational Mathematics and Scientific/Engineering Computing,\\
Academy of Mathematics and Systems Science,\\
Chinese Academy of Sciences}
\date[\textcolor{white} ]
{December 20, 2022}

%------------------------------------------------------------
%This block of commands puts the table of contents at the 
%beginning of each section and highlights the current section:
%\AtBeginSection[]
%{
%  \begin{frame}
%    \frametitle{Contents}
%    \tableofcontents[currentsection]
%  \end{frame}
%}
\AtBeginSection[]{
  \begin{frame}
  \vfill
  \centering
  \begin{beamercolorbox}[sep=8pt,center,shadow=true,rounded=true]{title}
    \usebeamerfont{title}\insertsectionhead\par%
  \end{beamercolorbox}
  \vfill
  \end{frame}
}
%------------------------------------------------------------
\AtBeginSubsection[]{
    \begin{frame}
    \vfill
    \centering
    \begin{beamercolorbox}[sep=8pt,center,shadow=true,rounded=true]{title}
        \usebeamerfont{title}\insertsubsectionhead\par%
    \end{beamercolorbox}
    \vfill
    \end{frame}
}

\begin{document}

%The next statement creates the title page.
\frame{\titlepage}
\begin{frame}
\frametitle{Contents}
\tableofcontents
\end{frame}


\section{Introduction}
\begin{frame}
    \frametitle{Background}
        \begin{itemize}
            \item \textbf{Weapon Target Assignment problem}is a problem in the military field
            \item The basic problem is to consider using m weapons to attack n targets, in order to minimize the weighted survival probability of all targets.
            \item \textbf{The problem is proved to be NP-complete}, and is often formulated as a nonlinear model.
            \item Before 2015, it took 16.2 hours to compute on a problem with 80 weapons and 80 targets.
        \end{itemize}
\end{frame}

\begin{frame}
    \frametitle{History}
    \begin{itemize}
        \item Give the basic formulation of WTA Problem (1958)
        \item linear approximations to problems, solve simplified forms of problems, or small-scale problems.
        \item Give the dynamic WTA Problem (1985).
        \item Improve the model of static and dynamic WTA problems and use heuristic algorithms to solve them.
        \item equivalent linear formulations of the WTA problem and solve it exactly.
    \end{itemize}
\end{frame}

\begin{frame}
    \frametitle{Basic formulation}
    \begin{columns}
        \begin{column}{.50\linewidth}
            \footnotesize
            \begin{itemize}
                \item $I = \left\{1,\cdots,m\right\} $, Weapon set.
                \item $J = \left\{1,\cdots,n\right\} $, Target set.
                \item $p_{ij}\in [0,1]$, probability that $i$ hits $j$
            \end{itemize}
        \end{column}
    \hspace{-1cm}
        \begin{column}{.50\linewidth}
            \footnotesize
            \begin{itemize}
                \item $V_j$, The weight of the target $j$.
                \item $x_{ij}$, whether weapon $i$ attack $j$.
            \end{itemize}
        \end{column}
    \end{columns}
    
    \begin{align*} \tag{S0}
        \max\quad & \sum_{j=1}^n (1 - V_j) \left( 1 - \prod_{i=1}^m (1 -  p_{ij})^{x_{ij}} \right) \\ 
        \mathrm{s. t.}\quad &\sum_{j=1}^n x_{ij} \leq 1\quad \forall ~i \in I,\\
        & x_{ij} \in \left\{ 0,1 \right\} \quad \forall~ j\in J , ~ i \in I.
    \end{align*}
\end{frame}


\begin{frame}
    \frametitle{Basic formulation}
    \begin{columns}
        \begin{column}{.50\linewidth}
            \footnotesize
            \begin{itemize}
                \item $I = \left\{1,\cdots,m\right\} $, Weapon set.
                \item $J = \left\{1,\cdots,n\right\} $, Target set.
                \item $p_{ij}\in [0,1]$, probability that $i$ hits $j$
            \end{itemize}
        \end{column}
    \hspace{-1cm}
        \begin{column}{.50\linewidth}
            \footnotesize
            \begin{itemize}
                \item $V_j$, The weight of the target $j$.
                \item $x_{ij}$, whether weapon $i$ attack $j$.
            \end{itemize}
        \end{column}
    \end{columns}
    
    \begin{align}
        \min\quad & \sum_{j=1}^n V_j \left( \prod_{i=1}^m (1 -  p_{ij})^{x_{ij}} \right) \\ 
        \mathrm{s. t.}\quad &\sum_{j=1}^n x_{ij} \leq 1\quad \forall ~i \in I,\\
        & x_{ij} \in \left\{ 0,1 \right\} \quad \forall~ j\in J , ~ i \in I.
    \end{align}
\end{frame}



\section{Formulations and Algorithms}

\begin{frame}
    \frametitle{Category}
    \begin{itemize}
        \item Static WTA problems.
        \item Dynamic WTA problems
        \item In recent years, some new problem forms have also be propose. such as adding more combat elements or using multi-objective programs.
    \end{itemize}
\end{frame}


\begin{frame}
    \frametitle{Notation}
    \begin{columns}
        \begin{column}{.50\linewidth}
            \footnotesize
            \begin{itemize}
                \item $I$, weapon set.
                \item $J$, target set.
                \item $n$, the number of targets.
                \item $m$, the number of weapon types. If limiting to one of each type, this variable indicates the number of weapons.
                \item $w_i$, the number of class i weapon
            \end{itemize}
        \end{column}
        \begin{column}{.50\linewidth}
            \footnotesize
            \begin{itemize}
                \item $p_{ij}$, The probability that weapon i hits target j.
                \item $q_{ij}$, The probability that weapon i misses target j.
                \item $x_{ij}$, The number of weapons of type i assigned to target j. If each weapon is limited to only one, this variable is a 0-1 variable.
                \item $V_j$, value of target j.
                \item $c_{ij}$, The cost of assigning weapon i to target j
            \end{itemize}
        \end{column}
    \end{columns}
\end{frame}

\begin{frame}
    \frametitle{static WTA model 1}
    \begin{itemize}
        \item Compared to the basic model, allows $w_i$ weapons for each weapon type.
    \end{itemize}
    \begin{align*} \tag{S1.1}
        \min\quad & \sum_{j=1}^n V_j \left( \prod_{i=1}^m (1 -  p_{ij})^{x_{ij}} \right) \\ 
        \mathrm{s. t.}\quad &\sum_{j=1}^n x_{ij} \leq w_i\quad \forall ~i \in I,\\
        & x_{ij} \in \mathbb{Z}^+ \quad \forall~ j\in J , ~ i \in I.
    \end{align*}
    
\end{frame}
\begin{frame}
    \frametitle{static WTA model 1}
    \begin{itemize}
        \item Transform $1-p_{ij}$ to $q_{ij}$ to make the formulation simpler.
    \end{itemize}
    \begin{align*} \tag{S1.2}
        \min\quad & \sum_{j=1}^n V_j \left( \prod_{i=1}^m q_j^{x_{ij}} \right) \\ 
        \mathrm{s. t.}\quad &\sum_{j=1}^n x_{ij} \leq w_i\quad \forall ~i \in I,\\
        & x_{ij} \in \mathbb{Z}^+ \quad \forall~ j\in J , ~ i \in I.
    \end{align*}
\end{frame}

\begin{frame}
    \frametitle{static WTA model 2}
    \begin{itemize}
        \item For Convenience,Assume that any weapon has the same probability of  hitting the same target.
    \end{itemize}
    \begin{align*} \tag{S2}
        \min\quad & \sum_{j=1}^n V_j q_j^{\sum_{i=1}^m x_{ij}} \\ 
        \mathrm{s. t.}\quad &\sum_{j=1}^n x_{ij} \leq w_i\quad \forall ~i \in I,\\
        & x_{ij} \in \mathbb{Z}^+ \quad \forall~ j\in J , ~ i \in I.
    \end{align*}
\end{frame}

\begin{frame}
    \frametitle{static WTA model 3}
    \begin{itemize}
        \item Logarithm of the objective function in S1
    \end{itemize}
    \begin{align*} \tag{S3.1}
        \min\quad & \sum_{j=1}^n V_j \exp \left( \sum_{i=1}^m {x_{ij}\ln(1 -  p_{ij})} \right) \\ 
        \mathrm{s. t.}\quad &\sum_{j=1}^n x_{ij} \leq w_i\quad \forall ~i \in I,\\
        & x_{ij} \in \mathbb{Z}^+ \quad \forall~ j\in J , ~ i \in I.
    \end{align*}
\end{frame}

\begin{frame}
    \frametitle{static WTA model 3}
    \begin{itemize}
        \item replaced the objective function with $y_j$ to get the following model.
    \end{itemize}
    \begin{align*} \tag{S3.2}
        \min\quad & \sum_{j=1}^n V_j e^{y_j}  \\ 
        \mathrm{s. t.}\quad &\sum_{j=1}^n x_{ij} \leq w_i\quad \forall ~i \in I,\\
        & \sum_{i=1}^m {\ln(1 -  p_{ij}) x_{ij}} = y_j \quad \forall ~ j \in J\\
        & x_{ij} \in \mathbb{Z}^+ \quad \forall~ j\in J , ~ i \in I.
    \end{align*}
    \begin{itemize}
        \item The model is equivalent to model S1, but it is more convenient for the solver to calculate.
        \item \textcolor{blue}{Kline et al.(2017b)} points out that use the commercial solver BARON to solve the problem, this form can increase the correct rate by 21\%
    \end{itemize}
\end{frame}



\begin{frame}
    \frametitle{static WTA model 4}
    \begin{itemize}
        \item limit the $x_{ij}$ to bianry variable.
        \item For $(1-p_{ij}x_{ij}) = (1-p_{ij})^{x_{ij}},~ x \in \left\{ 0,1\right\}$,The problem can be transformed into the following form.
    \end{itemize}
    \begin{align*} \tag{S4}
        \min\quad & \sum_{j=1}^n V_j \left( \prod_{i=1}^m (1 -  p_{ij})^{x_{ij}} \right) \\ 
        \mathrm{s. t.}\quad &\sum_{j=1}^n x_{ij} \leq 1\quad \forall ~i \in I,\\
        & x_{ij} \in \left\{ 0,1 \right\} \quad \forall~ j\in J , ~ i \in I.
    \end{align*}
    \begin{itemize}
        \item This conversion will cause the relaxation of the problem from a convex from to a non-convex form.
    \end{itemize}
\end{frame}

\begin{frame}
    \frametitle{static WTA model 5}
    \begin{itemize}
        \item Transform the problem into a knapsack problem
    \end{itemize}
    \begin{align*} \tag{S5}
        \min\quad & \sum_{j=1}^n{\sum_{x=1}^m{c_{ij}x_{ij}}} \\ 
        \mathrm{s. t.}\quad &\sum_{j=1}^n x_{ij} \leq 1\quad \forall ~i \in I,\\
        & \sum_{i=1}^m x_{ij} \leq 1\quad \forall ~j \in J \\
        & x_{ij} \in \left\{ 0,1 \right\} \quad \forall~ j\in J , ~ i \in I.
    \end{align*}
    \begin{itemize}
        \item This method removes the coupling when multiple weapons strike the same weapon, transforms the difficult objective function into a linear function.
    \end{itemize}
\end{frame}

\begin{frame}
    \frametitle{static WTA model 6}
    \begin{itemize}
        \item This is attacking model.
        \item there are K assets we have to protect, and n targets try to destroy these assets. 
        \item The probablitiy with target j will destroy k is $\gamma_{jk}$. 
    \end{itemize}
    \begin{align*} \tag{S6}
        \min\quad & \sum_{k=1}{K}a_k\prod_{j=1}^{n_k}\left[ \gamma_{ij} \prod_{i=1}^m (1 -  p_{ij})^{x_{ij}}\right]\\
        \mathrm{s. t.}\quad &\sum_{j=1}^n x_{ij} \leq 1\quad \forall ~i \in I,\\
        & \sum_{k=1}^{K}n_k = n\\
        & x_{ij} \in \left\{ 0,1 \right\} \quad \forall~ j\in J , ~ i \in I.
    \end{align*}
\end{frame}

\begin{frame}
    \frametitle{dynamic WTA model 1 : shoot-look-shoot}
    \begin{itemize}
        \item the model assumes that it is possible to observe the target situationn after a strike and rearrange weapons on targets that have not yet been destroyed
        \item In this model, no new targets will appear in the new time stage.
        \item Repeatedly attack the targets until all targets are destroyed or the number of iterations reaches the upper limit
    \end{itemize}
\end{frame}

\begin{frame}
    \frametitle{dynamic WTA model 2 : multi-stages}
    \begin{itemize}
        \item The model assumes that targets emergence will occur in different stages
        \item the given number of targets is engaged in first stage, the new targets that appear in later stages are given in the form of probability distributions
        \item Due to the high complexity of the model, heuristic solutions are mainly used.
    \end{itemize}
\end{frame}


\begin{frame}
    \frametitle{Sensor WTA problems}
    \begin{itemize}
        \item Considering the real scene, it is necessary to introduce sensors to identify targets to be attacked, and similarly, elements such as command posts also need to be introduced.
        \item The following is a simpler model: $S$ represents the set of sensors, and $o$ represents the number of sensors
    \end{itemize}
    \begin{align*} \tag{Sensor}
        \min\quad & \sum_{j=1}^n V_j \left( \prod_{i=1}^m {\prod_{s = 1}^{o} (1 -  p_{isj})^{x_{isj}}} \right) \\ 
        \mathrm{s. t.}\quad &\sum_{j=1}^n {\sum_{s=1}^o x_{isj}} \leq 1\quad \forall ~i \in I,\\
        &\sum_{j=1}^n {\sum_{i=1}^m x_{isj}} \leq 1\quad \forall ~s \in S,\\
        & x_{isj} \in \left\{ 0,1 \right\} \quad \forall~ j\in J , ~ i \in I,~ s\in S.
    \end{align*}
    \begin{itemize}
        \item The two constraints are each weapon and each sensor can only be assigned to one target.
    \end{itemize}
\end{frame}

\begin{frame}
    \frametitle{Multi-objective programs}
    \begin{itemize}
        \item In the real scene, not only the probability of destroying the target needs to be considered
        \item Introduced objectives include minimizing cost or minimizing time
        \item minimizing cost: Using weapon $i$ on target $j$ requires a fixed cost
        \item minimizing time: minimize combat time in order to achieve combat objectives
        \item combine multiple goals or Simultaneously optimize multiple objectives.
    \end{itemize}
\end{frame}



\section{Outer Approximation}
\begin{frame}
    \frametitle{basic formulation}
    \begin{columns}
        \begin{column}{.50\linewidth}
            \footnotesize
            \begin{itemize}
                \item $I = \left\{1,\cdots,m\right\} $, Weapon set.
                \item $J = \left\{1,\cdots,n\right\} $, Target set.
                \item $p_{ij}\in [0,1]$, probability that $i$ hits $j$
            \end{itemize}
        \end{column}
    \hspace{-1cm}
        \begin{column}{.50\linewidth}
            \footnotesize
            \begin{itemize}
                \item $V_j$, The weight of the target $j$.
                \item $x_{ij}$, whether weapon $i$ attack $j$.
            \end{itemize}
        \end{column}
    \end{columns}
    
    \begin{align*} \tag{S0}
        \min\quad & \sum_{j=1}^n a_j \left( \prod_{i=1}^m (1 -  p_{ij})^{x_{ij}} \right) \\ 
        \mathrm{s. t.}\quad &\sum_{j=1}^n x_{ij} \leq 1\quad \forall ~i \in I,\\
        & x_{ij} \in \left\{ 0,1 \right\} \quad \forall~ j\in J , ~ i \in I.
    \end{align*}
\end{frame}

\begin{frame}
    \frametitle{formulation}
    \begin{equation*}
        \prod_{i=1}^{m}{(1-p_{ij}x_{ij})} \iff \prod_{i=1}^{m}{(1-p_{ij})^{x_{ij}}}, x \in \left\{ 0,1 \right\}
    \end{equation*}
    \begin{itemize}
        \item When restricting $x$ to be a bianry variable, the objective function values are the same.
    \end{itemize}
\end{frame}

\begin{frame}
    \frametitle{The convexity of the objective function}
    consider the second formulation
    \begin{equation*}
        f_j(x) =  \prod_{i=1}^m (1 -  p_{ij})^{x_{ij}}\quad \forall~  j \in J
    \end{equation*}
    Hessian matrix$H$:
    \begin{equation*}
        \frac{\partial f_j(x)}{\partial x_{aj} \partial x_{bj}} = \ln(1 - p_{aj}) \ln(1 - p_{bj}) f_j(x)
    \end{equation*}
    {
    \scriptsize
    \begin{equation*}
        H = f(x)\begin{bmatrix}
            \ln(1 - p_{1j}) \ln(1 - p_{1j}) & \ln(1 - p_{1j}) \ln(1 - p_{2j}) &\cdots & \ln(1 - p_{1j}) \ln(1 - p_{mj}) \\
            \ln(1 - p_{2j}) \ln(1 - p_{1j}) & \ln(1 - p_{2j}) \ln(1 - p_{2j}) &\cdots & \ln(1 - p_{2j}) \ln(1 - p_{mj}) \\
            \vdots & \vdots &\ddots &\vdots \\
            \ln(1 - p_{mj}) \ln(1 - p_{1j}) & \ln(1 - p_{mj}) \ln(1 - p_{2j}) &\cdots & \ln(1 - p_{mj}) \ln(1 - p_{mj})


        \end{bmatrix}
    \end{equation*}
    }
\end{frame}

\begin{frame}
    \frametitle{The convexity of the objective function}
    we assume that:
    \begin{equation*}
        l = \begin{bmatrix}
            \ln(1 - p_{1j}) & \ln(1 - p_{2j}) & \cdots & \ln(1 - p_{mj})
        \end{bmatrix}
    \end{equation*}
    the Hessian matrix is :
    \begin{equation*}
        H = f(x)l\cdot l^T
    \end{equation*}
    It can be seen that the matrix is a rank-one matrix, that is, the columns are linearly related, so the determinant is 0 so the original function is convex.
\end{frame}

\begin{frame}
    \frametitle{transformed model}
    objective function
    \begin{equation*}
        \sum_{j=1}^n a_j \left( \prod_{i=1}^m (1 -  p_{ij})^{x_{ij}} \right)
    \end{equation*}
    {\footnotesize
    by introducing auxiliary variables, the nonlinear term can be transformed from the objective function into the constraint.
    
    if we take $\eta_j$ as the auxiliary variables to $\prod_{i=1}^m (1 -  p_{ij})^{x_{ij}}$ the model can be transformed to 
    \begin{align*} \tag{S0'}
        \min\quad & \sum_{j=1}^n a_j \eta_j \\ 
        \mathrm{s. t.}\quad & \eta_j \geq \prod_{i=1}^m (1 -  p_{ij})^{x_{ij}}, \quad \forall ~ j \in J \\ 
        &\sum_{j=1}^n x_{ij} \leq 1\quad \forall ~ i \in I,\\
        & x_{ij} \in \left\{ 0,1 \right\} \quad \forall ~ j\in J\quad i \in I.
    \end{align*}
    }
\end{frame}

\begin{frame}
    \frametitle{Basic idea of outer approximation}
    If $f(x)$ is a convex function, then for any point $x^*$ in the feasible region, we have
    \begin{equation*}
        f(x) \geq f(x^*) + \nabla f(x^*)(x - x^*)
    \end{equation*}
    Therefore, if the constraint of the original problem is $\eta \geq f(x)$, then
    \begin{equation*}
        \eta \geq f(x) \geq f(x^*) + \nabla f(x^*)(x - x^*)
    \end{equation*}
    must be correct
    \begin{itemize}
        \item That is, for any given point, a linear constraint can be introduced to ensure that the feasible region satisfies the constraint.
        \item The outer approximation method hopes to take advantage of the integer characteristics and replace a nonlinear constraint by introducing multiple linear constraints, but guarantees that the problems before and after the transformation are equivalent.
    \end{itemize}
    
    
\end{frame}

\begin{frame}
    \frametitle{constraint formulation of outer approximation}
    take $f_j(x) =  \prod_{i=1}^m (1 -  p_{ij})^{x_{ij}}\quad j \in J$,then for any given  $\bar{x}$ in feasible domain, we have 
    \begin{align*}
        \nabla f(\bar{x})(x - \bar{x}) & = f(\bar{x})\sum_{i = 1}^m \ln(1-p_{ij})(x_{ij} - \bar{x_{ij}})\\
        &= f(\bar{x})\sum_{i = 1}^m \ln(1-p_{ij})x_{ij} - f(\bar{x})\sum_{i = 1}^m \ln(1-p_{ij})\bar{x_{ij}}
    \end{align*}
    \begin{itemize}
        \item The first term is the linear combination of variables, and the second term is a constant when $\bar{x}$ is given.
        \item We denote this constraint as the outer approximation constraint.
    \end{itemize}
    

    
\end{frame}

\begin{frame}
    \frametitle{outer approximation model}
    
    {\scriptsize
    Let $X$ donates the set of all integer feasible solutions in the problem, the model can be described as.
    \begin{align*} \tag{OA}
        \min\quad & \sum_{j=1}^n a_j \eta_j \\ 
        \mathrm{s. t.}\quad & \eta_j \geq f(\bar{x})\sum_{i = 1}^m \ln(1-p_{ij})(x_{ij} - \bar{x_{ij}}) + f(\bar{x}), \quad \forall ~ j \in J,\ \bar{x} \in X \\ 
        &\sum_{i=1}^m x_{ij} \leq 1\quad \forall ~ j \in J,\\
        & x_{ij} \in \left\{ 0,1 \right\} \quad \forall ~ j\in J\quad i \in I.
    \end{align*}
    the number of outer approximation constraints is very large, which has caused certain difficulties in solving the problem. Therefore, the restricted model is considered, that is, only some constraints are considered.
    }
   
\end{frame}

\begin{frame}
    \frametitle{Idea of outer approximation method}
    {
    \scriptsize
    There are too many outer approximation constraints, only a subset of them is considered : $\hat{X} \subseteq X$
    \begin{align*}
        \min\quad & \sum_{j=1}^n a_j \eta_j \\ 
        \mathrm{s. t.}\quad & \eta_j \geq f(\bar{x})\sum_{i = 1}^m \ln(1-p_{ij})(x_{ij} - \bar{x_{ij}}) + f(\bar{x}), \quad j \in J,\ \bar{x} \in X \\ 
        &\sum_{i=1}^m x_{ij} \leq 1\quad j \in J,\quad x_{ij} \in \left\{ 0,1 \right\} \quad j\in J\quad i \in I
    \end{align*}
    \begin{enumerate}
        \item Remove all outer approximation constraints and give the current optimal solution $x, \eta$
        \item Check whether the current optimal solution can satisfy all the outer approximation constraints. if true, the iteration terminates and end the execution.
        \item Otherwise, choose to outer approximation constraint that has been violate most and add it to the restricted model, resolve the model and go to step 2.
    \end{enumerate}
    }
\end{frame}

\begin{frame}
    \frametitle{choose violate constraint}
    {
    \scriptsize
    \begin{align*}
        \min\quad & \sum_{j=1}^n a_j \eta_j \\ 
        \mathrm{s. t.}\quad & \eta_j \geq f(\bar{x})\sum_{i = 1}^m \ln(1-p_{ij})(x_{ij} - \bar{x_{ij}}) + f(\bar{x}), \quad j \in J,\ \bar{x} \in X \\ 
        &\sum_{i=1}^m x_{ij} \leq 1\quad j \in J,\quad x_{ij} \in \left\{ 0,1 \right\} \quad j\in J\quad i \in I
    \end{align*}
    \begin{itemize}
        \item Unless an optimal solution has been found, the solution to the restricted problem must bring a violation of the outer approximation constraint.
        \item It can be intuitively understood that a violation means that $\eta_j$ is outside the feasible region.
        \item When obtaining a solution that $x$ is in the feasible region, if $\eta < f_j(x)$, this causes infeasible. the infeasible point can be excluded by adding outer approximation constraint.
    \end{itemize}
    }
\end{frame}


\section{Columnn Generation}
\begin{frame}
    \frametitle{Basic Idea}
    \begin{itemize}
        \item Some linear programming problems have too many columns (variables), making it difficult to solve
        \item Use only some of the variables at the beginning of the algorithm and assume all the other variables are 0
        \item Variables that have the potential to improve the objective function are iteratively added to the model. 
        \item Once it can be proven that adding new variables will no longer improve the value of the objective function, the iterative process is terminated and an optimal solution is obtained.
    \end{itemize}
\end{frame}

\begin{frame}
    \frametitle{How to use column generation in WTA Problem}
    \begin{itemize}
		\item \textbf{Basic Idea}:By listing all the weapon assignment scenarios $S$, the problem is transformed into a linear programming problem.
		\item Assuming that there are $m$ weapons, for any target $j$, each weapon can choose to attack $j$ or not attack $j$, so there are $2^m$ different attack schemes.
		\item \textbf{an example}:There are a total of 8 weapons, and the attack plan using No. 1, 3, and 6 weapons is recorded as $s_{[1,0,1,0,0,1,0,0]}$, $|S| = 2^8 = 256$
		\item $n_{si}$: bianry variable, indicates whether to enable weapon $i$ in the $s$th scene. In the above example, $n_{s1} = 1,\ n_{s2} = 0$
		\item $q_{js} = a_j \prod_{i = 1}^m (1 - n_{si}\cdot p_{ij})$:weighted probability of the plan $s$ to hit the target $j$,For example, $q_{3s}$ is to use the No. 1, 3, and 6 weapons to hit the target $3$ and multiply the probability of destroying the target $j$ by the weight of the target $j$.
	\end{itemize}
\end{frame}

\begin{frame}
    \frametitle{transformed formulation}
    \optimalProblem{\min}{\sumFromTo{j = 1}{n}{\sumFromTo{s = 0}{2^m -1}{q_{js}y_{js}}}\tag{CG}}{\sumFromTo{j = 1}{n}{\sumFromTo{s = 0}{2^m -1}{n_{si}y_{js}}}\leq 1 \quad &\forall ~ i \in I\\& \sumFromTo{s = 0}{2^m - 1}{y_{js}} = 1 \quad &\forall ~ j \in J\\& y_{js} \in \left\{ 0,1 \right\} &\forall ~ j\in J,\ s\in S}
    \begin{columns}
        \begin{column}{.55\linewidth}
            \footnotesize
            \begin{itemize}
                \item $I = \left\{1,\cdots,m\right\} $, weapon set
                \item $J = \left\{1,\cdots,n\right\} $, target set
                \item $S = \left\{1,\cdots,2^m\right\}$, scene set
            \end{itemize}
        \end{column}
    \hspace{-3cm}
        \begin{column}{.55\linewidth}
            \footnotesize
            \begin{itemize}
                \item $n_{si}$:weather scene $s$ use weapon $i$
                \item $y_{js}$:Whether use $s$ to arrack target $j$
                \item $q_{js}$:weighted destruction probability of using $s$ to $j$
            \end{itemize}
        \end{column}
    \end{columns}
\end{frame}

\begin{frame}
    \frametitle{transformed formulation continuous}
    \optimalProblem{\min}{\sumFromTo{j = 1}{n}{\sumFromTo{s = 0}{2^m -1}{q_{js}y_{js}}}\tag{CG}}{\sumFromTo{j = 1}{n}{\sumFromTo{s = 0}{2^m -1}{n_{si}y_{js}}}\leq 1 \quad &\forall ~ i \in I\\& \sumFromTo{s = 0}{2^m - 1}{y_{js}} = 1 \quad &\forall ~ j \in J\\& y_{js} \in \left\{ 0,1 \right\} & \forall ~ j\in J,\ s\in S}
    \begin{itemize}
        \item \textbf{objective function}:minimize the weighted destruction probabilities.
        \item \textbf{first constraint}:Each weapon can only attack one target.
        \item \textbf{second constraint}:assignn exactly one scene for each target
    \end{itemize}
\end{frame}

\begin{frame}
    \frametitle{column enumeration}
    \optimalProblem{\min}{\sumFromTo{j = 1}{n}{\sumFromTo{s = 0}{2^m -1}{q_{js}y_{js}}}\tag{CG}}{\sumFromTo{j = 1}{n}{\sumFromTo{s = 0}{2^m -1}{n_{si}y_{js}}}\leq 1 \quad &\forall ~ i \in I\\& \sumFromTo{s = 0}{2^m - 1}{y_{js}} = 1 \quad &\forall ~ j \in J\\& y_{js} \in \left\{ 0,1 \right\} & \forall ~ j\in J,\ s\in S}

    \begin{itemize}
        \item This idea is from \textcolor{blue}{Lu(2021)}
        \item The basic idea of the column enumeration method is to enumerate all columns in a smarter way
        \item Two techniques are used in the article: weapon number bounding and weapon domination
        \item weapon number bounding:Scenarios with too few or too many weapons are give no improvement to the objective function.
        \item weapon domination:For a certain target, it is better to arrange one weapon than another
    \end{itemize}
\end{frame}




\begin{frame}
    \frametitle{LP Relaxition}
    \optimalProblem{\min}{\sumFromTo{j = 1}{n}{\sumFromTo{s = 0}{2^m -1}{q_{js}y_{js}}}\tag{CG-LP}}{\sumFromTo{j = 1}{n}{\sumFromTo{s = 0}{2^m -1}{n_{si}y_{js}}}\leq 1 \quad &\forall ~ i \in I\\& \sumFromTo{s = 0}{2^m - 1}{y_{js}} = 1 \quad &\forall ~  j \in J\\& y_{js} \geq 0 &\forall ~ j\in J,\ s\in S}
    \begin{itemize}
        \item LP Relaxition, the third constraint can be changed into $y_{js} \geq 0$
        \item linear program contains $n \times 2^m$ columns
        \item Only some of the columns (variables) are taken, because in the optimal solution of linear programming, at most $m+n$ variables are not 0, that is, the dual constraints corresponding to other variables are not active.
    \end{itemize}
\end{frame}

\begin{frame}
    \frametitle{Dual Problem}
    
    \optimalProblem{\max}{\sumFromTo{i=1}{m}{u_i} + \sumFromTo{j=1}{n}{v_j}\tag{CG-Dual}}{\sumFromTo{i=1}{m}{x_{is}u_i}+ v_j \leq q_{js}\quad \forall ~(s,j) \in X\\ &u_i \leq 0,\quad v_j\ \mathrm{free}}
    \begin{itemize}
        \item If $X = \left\{(s,j) | s\in S , j \in J \right\}$, each element in $X$ corresponds to a constraint.
        \item \item Selecting some columns in the original problem is equivalent to selecting some rows in the dual problem.
    \end{itemize}
\end{frame}

\begin{frame}
    \frametitle{Restricted Dual Problem}
    
    \optimalProblem{\max}{\sumFromTo{i=1}{m}{u_i} + \sumFromTo{j=1}{n}{v_j}\tag{CG-Dual-R}}{\sumFromTo{i=1}{m}{x_{is}u_i}+ v_j \leq q_{js}\quad ~\forall (s,j) \in \hat{X} \\ &u_i \leq 0,\quad v_j\ \mathrm{free}}
    \begin{itemize}
        
        \item Only select some constraints, that is, only consider the constraints generated by the some $(s,j)$ in $\hat{X} \subseteq X$
        \item Solve to get $u^*, v^*$ and bring into the original dual, if all the constraints are satisfied, it must be the optimal solution.
        \item Otherwise, choose the constraint that violates the most, that is, the largest $\sumFromTo{i=1}{m}{x_{is}u_i}+ v_j > q_{js}$.
    \end{itemize}
\end{frame}

\begin{frame}
    \frametitle{subproblem}
    \optimalProblem{\min}{a_j \prod_{i=1}^{m}{(1-p_{ij}x_{is})} -\sumFromTo{i=1}{m}{x_{is}}u_i - v_j}{j \in J,\quad s \in S\tag{CG-sub}}
    \begin{itemize}
        \item choose the constraint that violates the most, that is, the largest $\sumFromTo{i=1}{m}{x_{is}u_i}+ v_j > q_{js}$.
        \item using the definition of $q_{js}$ to get the above sub-problems.
        \item The problem can be separated and then solved for each given target $j$.
        \item It is intuitive that using each weapon to attack requires a cost, we try  to balance the cost of the weapon and the probability of destroying the target.
    \end{itemize}
\end{frame}

\begin{frame}
    \frametitle{motivation to use column generation}
    \begin{itemize}
        \item Huge improvement in computational result for column enumerations
        \item The article on column enumeration mentions that column generation subproblems is hard to solve due to non-linearity
        \item The outer approximation method can be used in subproblems
    \end{itemize}
\end{frame}

\section{Future Research Plan}
\begin{frame}
    \frametitle{Future Research Plan}
    \begin{itemize}
        \item Implement the column generation method.
        \item Add sensors to problem formulation.
    \end{itemize}
\end{frame}

\begin{frame}
    \frametitle{Alternate applications}
    \begin{align}
        \min\quad & \sum_{j=1}^n V_j \left( \prod_{i=1}^m (1 -  p_{ij})^{x_{ij}} \right) \\ 
        \mathrm{s. t.}\quad &\sum_{j=1}^n x_{ij} \leq 1\quad \forall ~i \in I,\\
        & x_{ij} \in \left\{ 0,1 \right\} \quad \forall~ j\in J , ~ i \in I.
    \end{align}
    \begin{itemize}
        \item This model can not only be applied to the problem of WTA, but also some assignment problems with limited resources.
    \end{itemize}
\end{frame}

\begin{frame}
\textcolor{myNewColorA}{\Huge{\centerline{Thank you!}}}
\end{frame}


\end{document}