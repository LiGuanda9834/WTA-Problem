\documentclass[CJK,10pt]{beamer}
%\usepackage[utf8]{inputenc}
\usepackage{xeCJK}
\usepackage{graphicx}
\usepackage{subcaption}
\usepackage {mathtools}
%\usepackage{utopia} %font utopia imported
\usetheme{CambridgeUS}
\usecolortheme{dolphin}

% set colors
\definecolor{myNewColorA}{RGB}{25,25,112}
\definecolor{myNewColorB}{RGB}{25,25,112}
\definecolor{myNewColorC}{RGB}{25,25,112}
\setbeamercolor*{palette primary}{bg=myNewColorC}
\setbeamercolor*{palette secondary}{bg=myNewColorB, fg = white}
\setbeamercolor*{palette tertiary}{bg=myNewColorA, fg = white}
\setbeamercolor*{titlelike}{fg=myNewColorA}
\setbeamercolor*{title}{bg=myNewColorA, fg = white}
\setbeamercolor*{item}{fg=myNewColorA}
\setbeamercolor*{caption name}{fg=myNewColorA}
\usefonttheme{professionalfonts}
\usepackage{natbib}
\usepackage{hyperref}
%------------------------------------------------------------
%\titlegraphic{\includegraphics[height=1.5cm]{logo.png}} 

\setbeamerfont{title}{size=\large}
\setbeamerfont{subtitle}{size=\small}
\setbeamerfont{author}{size=\small}
\setbeamerfont{date}{size=\small}
\setbeamerfont{institute}{size=\small}



%设置页码脚标
%\setbeamertemplate{footline}[frame number]
\defbeamertemplate{footline}{NGEGFootlineTemplate}{%
	\leavevmode% 离开vmode,也就是离开竖直模式,进入水平模式
	\hbox{%
		\begin{beamercolorbox}[wd=.2\paperwidth,ht=2.25ex,dp=1ex,center]{author in head/foot}%
			\ifnum \the\value{page}>1 \usebeamerfont{author in head/foot}\insertshortauthor\fi
		\end{beamercolorbox}%
		\begin{beamercolorbox}[wd=.6\paperwidth,ht=2.25ex,dp=1ex,center]{title in head/foot}%
			\ifnum \the\value{page}>1 \usebeamerfont{title in head/foot}\insertshorttitle\fi
		\end{beamercolorbox}%
		\begin{beamercolorbox}[wd=0.2\paperwidth,ht=2.25ex,dp=1ex,center]{author in head/foot}%
			\ifnum \the\value{page}>1 \insertframenumber{} / \inserttotalframenumber\fi
	\end{beamercolorbox}}%
%	\vskip0pt%
}
\setbeamertemplate{footline}[NGEGFootlineTemplate]



\newcommand{\mb}[1]{\mathbf{#1}}
\newcommand{\mc}[1]{\mathcal{#1}}
\newcommand{\E}{\mathcal{E}}
\newcommand{\C}{\mathcal{C}}
\newcommand{\x}{\mathbf{x}}
\newcommand{\dd}{\mathrm{d}}

\newcommand{\sumFromTo}[3]{\ensuremath{\sum_{#1}^{#2} #3}}

\newcommand{\optimalProblem}[3]
{\begin{align*}
    #1 \quad &#2 \\
    \mathrm{s. t.}\quad&#3
\end{align*}}




\title[Weapon Target Assignment problem]{Literature Review of Weapon Target Assignment problem}%主标题
%\subtitle{ }%%副标题
\author[Guanda Li]{Guanda Li
\and joint work with Liang Chen and Yu-hong Dai}%%作者

\institute[LSEC]{Institute of Computational Mathematics and Scientific/Engineering Computing,\\
Academy of Mathematics and Systems Science,\\
Chinese Academy of Sciences}
\date[\textcolor{white} ]
{December 20, 2022}

%------------------------------------------------------------
%This block of commands puts the table of contents at the 
%beginning of each section and highlights the current section:
%\AtBeginSection[]
%{
%  \begin{frame}
%    \frametitle{Contents}
%    \tableofcontents[currentsection]
%  \end{frame}
%}
\AtBeginSection[]{
  \begin{frame}
  \vfill
  \centering
  \begin{beamercolorbox}[sep=8pt,center,shadow=true,rounded=true]{title}
    \usebeamerfont{title}\insertsectionhead\par%
  \end{beamercolorbox}
  \vfill
  \end{frame}
}
%------------------------------------------------------------
\AtBeginSubsection[]{
    \begin{frame}
    \vfill
    \centering
    \begin{beamercolorbox}[sep=8pt,center,shadow=true,rounded=true]{title}
        \usebeamerfont{title}\insertsubsectionhead\par%
    \end{beamercolorbox}
    \vfill
    \end{frame}
}

\begin{document}

%The next statement creates the title page.
\frame{\titlepage}
\begin{frame}
\frametitle{Contents}
\tableofcontents
\end{frame}


\section{Introduction}
\begin{frame}
    \frametitle{Background}
    \textbf{背景介绍}
        \begin{itemize}
            \item \textbf{Weapon Target Assignment problem}是一个军事领域中的问题
            \item 其基本问题是考虑使用m个武器攻击n个目标,以最小化所有目标的加权存活概率.{\tiny(或等价地,对目标造成的加权伤害最大化)}
            \item \textbf{该问题被证明是NP-C的},并且通常被表述为非线性模型。
            \item To best of my knowledge, 2015年前,在80个武器和80个目标的问题计算需要16.2个小时。
        \end{itemize}
\end{frame}

\begin{frame}
    \frametitle{history}
    \begin{itemize}
        \item 提出WTA问题形式Manne (1958)
        \item 通过对问题进行线性近似、求解问题的简化形式与求解小规模问题
        \item 首次提出动态WTA问题 (1985)
        \item 改进静态与动态WTA问题的模型,采用启发式算法进行求解
        \item 对问题进行等价的线性转换并精确求解
    \end{itemize}
\end{frame}

\begin{frame}
    \frametitle{basic formulation}
    \begin{columns}
        \begin{column}{.70\linewidth}
            \footnotesize
            \begin{itemize}
                \item $I = \left\{1,\cdots,m\right\} $, 武器集合.
                \item $J = \left\{1,\cdots,n\right\} $, 目标集合.
                \item $p_{ij}\in [0,1]$, $i$击落$j$的概率.
            \end{itemize}
        \end{column}
    \hspace{-1cm}
        \begin{column}{.35\linewidth}
            \footnotesize
            \begin{itemize}
                \item $V_j$, 目标$j$的权重.
                \item $x_{ij}$,武器$i$是否攻击$j$.
            \end{itemize}
        \end{column}
    \end{columns}
    
    \begin{align*} \tag{S0}
        \max\quad & \sum_{j=1}^n (1 - V_j) \left( \prod_{i=1}^m (1 -  p_{ij})^{x_{ij}} \right) \\ 
        \mathrm{s. t.}\quad &\sum_{j=1}^n x_{ij} \leq 1\quad \forall ~i \in I,\\
        & x_{ij} \in \left\{ 0,1 \right\} \quad \forall~ j\in J , ~ i \in I.
    \end{align*}
\end{frame}


\begin{frame}
    \frametitle{basic formulation}
    \begin{columns}
        \begin{column}{.70\linewidth}
            \footnotesize
            \begin{itemize}
                \item $I = \left\{1,\cdots,m\right\} $, 武器集合.
                \item $J = \left\{1,\cdots,n\right\} $, 目标集合.
                \item $p_{ij}\in [0,1]$, $i$击落$j$的概率.
            \end{itemize}
        \end{column}
    \hspace{-1cm}
        \begin{column}{.35\linewidth}
            \footnotesize
            \begin{itemize}
                \item $V_j$, 目标$j$的权重.
                \item $x_{ij}$,武器$i$是否攻击$j$.
            \end{itemize}
        \end{column}
    \end{columns}
    
    \begin{align}
        \min\quad & \sum_{j=1}^n V_j \left( \prod_{i=1}^m (1 -  p_{ij})^{x_{ij}} \right) \\ 
        \mathrm{s. t.}\quad &\sum_{j=1}^n x_{ij} \leq 1\quad \forall ~i \in I,\\
        & x_{ij} \in \left\{ 0,1 \right\} \quad \forall~ j\in J , ~ i \in I.
    \end{align}
\end{frame}



\section{Formulations and algorithms}

\begin{frame}
    \frametitle{whether the problem is static}
    \begin{itemize}
        \item WTA问题主要分为了静态WTA问题(SWTA问题)与动态WTA问题(DWTA问题)
        \item 静态WTA问题中不考虑不同的时间阶段,仅考虑针对一次目标来袭所做的拦截。
        \item 动态WTA问题则考虑多阶段的问题,在不同的阶段呢可能有不同的目标出现,武器的分配也同样需要照顾多个不同的时间阶段。
        \item 近年来,也有一些新的问题形式出现,比如加入更多的作战要素或者使用多目标规划。
    \end{itemize}
\end{frame}


\subsection{Static Weapon Target Assignment problem}


\begin{frame}
    \frametitle{Notation}
    \begin{columns}
        \begin{column}{.50\linewidth}
            \footnotesize
            \begin{itemize}
                \item $I$, Weapon set.
                \item $J$, Target set.
                \item $n$, 目标的个数。
                \item $m$, 武器种类的个数。如果限制每种武器只有一个,则该变量指示武器的个数。
                \item $w_i$, 第i类武器的个数
                \item $t$, 时间阶段(仅在DWTA问题中有效)
            \end{itemize}
        \end{column}
        \begin{column}{.50\linewidth}
            \footnotesize
            \begin{itemize}
                \item $p_{ij}$, 武器i摧毁目标j的概率。
                \item $q_{ij}$, 武器i未能摧毁目标j的概率。
                \item $x_{ij}$, 分配给目标j的种类为i的武器个数,如果限制每种武器仅有一个,则该变量为0-1变量。
                \item $V_j$, 目标j的摧毁价值。
                \item $c_{ij}$, 将武器i分配给目标j的花费
                \item $s_j$, 能够分配给目标j的最大武器数目
            \end{itemize}
        \end{column}
    \end{columns}
\end{frame}

\begin{frame}
    \frametitle{static WTA model 1}
    \begin{itemize}
        \item 相比于基础模型,允许每种武器有$w_i$个。
    \end{itemize}
    \begin{align*} \tag{S1.1}
        \min\quad & \sum_{j=1}^n V_j \left( \prod_{i=1}^m (1 -  p_{ij})^{x_{ij}} \right) \\ 
        \mathrm{s. t.}\quad &\sum_{j=1}^n x_{ij} \leq w_i\quad \forall ~i \in I,\\
        & x_{ij} \in \mathbb{Z}^+ \quad \forall~ j\in J , ~ i \in I.
    \end{align*}
    
\end{frame}
\begin{frame}
    \frametitle{static WTA model 1}
    \begin{itemize}
        \item 如果将$1-p_{ij}$记为$q_{ij}$,可以转化为看起来更简单的形式。
    \end{itemize}
    \begin{align*} \tag{S1.2}
        \min\quad & \sum_{j=1}^n V_j \left( \prod_{i=1}^m q_j^{x_{ij}} \right) \\ 
        \mathrm{s. t.}\quad &\sum_{j=1}^n x_{ij} \leq w_i\quad \forall ~i \in I,\\
        & x_{ij} \in \mathbb{Z}^+ \quad \forall~ j\in J , ~ i \in I.
    \end{align*}
\end{frame}

\begin{frame}
    \frametitle{static WTA model 2}
    \begin{itemize}
        \item 为了让计算更简便,假设任何武器攻击同一个目标的击毁概率都相同。
    \end{itemize}
    \begin{align*} \tag{S2}
        \min\quad & \sum_{j=1}^n V_j q_j^{\sum_{i=1}^m x_{ij}} \\ 
        \mathrm{s. t.}\quad &\sum_{j=1}^n x_{ij} \leq w_i\quad \forall ~i \in I,\\
        & x_{ij} \in \mathbb{Z}^+ \quad \forall~ j\in J , ~ i \in I.
    \end{align*}
\end{frame}

\begin{frame}
    \frametitle{static WTA model 3}
    \begin{itemize}
        \item 将S1中的目标函数取对数,即可得到如下模型
    \end{itemize}
    \begin{align*} \tag{S3.1}
        \min\quad & \sum_{j=1}^n V_j \exp \left( \sum_{i=1}^m {x_{ij}\ln(1 -  p_{ij})} \right) \\ 
        \mathrm{s. t.}\quad &\sum_{j=1}^n x_{ij} \leq w_i\quad \forall ~i \in I,\\
        & x_{ij} \in \mathbb{Z}^+ \quad \forall~ j\in J , ~ i \in I.
    \end{align*}
\end{frame}

\begin{frame}
    \frametitle{static WTA model 3}
    \begin{itemize}
        \item 将上述模型进行变量替换即可得到如下的模型
    \end{itemize}
    \begin{align*} \tag{S3.2}
        \min\quad & \sum_{j=1}^n V_j e^{y_j}  \\ 
        \mathrm{s. t.}\quad &\sum_{j=1}^n x_{ij} \leq w_i\quad \forall ~i \in I,\\
        & \sum_{i=1}^m {\ln(1 -  p_{ij}) x_{ij}} = y_j \quad \forall ~ j \in J\\
        & x_{ij} \in \mathbb{Z}^+ \quad \forall~ j\in J , ~ i \in I.
    \end{align*}
    \begin{itemize}
        \item 该模型从数学上与模型S1是等价的,但是新形式更方便求解器的求解。
        \item \textcolor{red}{Kline et al.(2017b)} 指出,使用商用求解器BARON求解问题,该形式能够让正确率提高21\%
    \end{itemize}
\end{frame}



\begin{frame}
    \frametitle{static WTA model 4}
    \begin{itemize}
        \item 该模型与一开始介绍的基本模型相同,即限制每种武器的类型只有一个。如果某个类型的武器有不止一个,可以将它们视为不同的。
        \item 由于$(1-p_{ij}x_{ij}) = (1-p_{ij})^{x_{ij}},~ x \in \left\{ 0,1\right\}$,可以将问题转化为如下形式。
    \end{itemize}
    \begin{align*} \tag{S4}
        \min\quad & \sum_{j=1}^n V_j \left( \prod_{i=1}^m (1 -  p_{ij})^{x_{ij}} \right) \\ 
        \mathrm{s. t.}\quad &\sum_{j=1}^n x_{ij} \leq 1\quad \forall ~i \in I,\\
        & x_{ij} \in \left\{ 0,1 \right\} \quad \forall~ j\in J , ~ i \in I.
    \end{align*}
    \begin{itemize}
        \item 该转化会导致问题的松弛从一个凸问题变成一个非凸问题,会导致一些凸的求解方法无法使用。
    \end{itemize}
\end{frame}

\begin{frame}
    \frametitle{static WTA model 5}
    \begin{itemize}
        \item 将问题转化为背包模型
    \end{itemize}
    \begin{align*} \tag{S5}
        \min\quad & \sum_{j=1}^n{\sum_{x=1}^m{c_{ij}x_{ij}}} \\ 
        \mathrm{s. t.}\quad &\sum_{j=1}^n x_{ij} \leq 1\quad \forall ~i \in I,\\
        & \sum_{i=1}^m x_{ij} \leq 1\quad \forall ~j \in J \\
        & x_{ij} \in \left\{ 0,1 \right\} \quad \forall~ j\in J , ~ i \in I.
    \end{align*}
    \begin{itemize}
        \item 该方法去掉了多个武器对同一个武器同时打击时的耦合,将困难的目标函数转化为了线性函数
        \item 虽然该方法能够比较快的求解,但是却不如之前的模型更贴近现实情况。
    \end{itemize}
\end{frame}

\begin{frame}
    \frametitle{static WTA model 6}
    \begin{itemize}
        \item This is attacking model.
        \item there are K assets we have to protect, and n targets try to destroy these assets. 
        \item The probablitiy with target j will destroy k is $\gamma_{jk}$. 
    \end{itemize}
    \begin{align*} \tag{S6}
        \min\quad & \sum_{k=1}{K}a_k\prod_{j=1}^{n_k}\left[ \gamma_{ij} \prod_{i=1}^m (1 -  p_{ij})^{x_{ij}}\right]\\
        \mathrm{s. t.}\quad &\sum_{j=1}^n x_{ij} \leq 1\quad \forall ~i \in I,\\
        & \sum_{k=1}^{K}n_k = n\\
        & x_{ij} \in \left\{ 0,1 \right\} \quad \forall~ j\in J , ~ i \in I.
    \end{align*}
\end{frame}

\subsection{Dynamic Weapon Target Assignment problem}
\begin{frame}
    \frametitle{dynamic WTA model 1 : shoot-look-shoot}
    \begin{itemize}
        \item 该模型假设可以在一次打击结束后对现状进行观察,对尚未被击毁的目标重新安排武器
        \item 在该模型中,不会有新的目标出现
        \item 将会对给定数量的目标进行反复攻击,直到全部目标都被摧毁或迭代次数到达上限
    \end{itemize}
\end{frame}

\begin{frame}
    \frametitle{dynamic WTA model 2 : multi-stages}
    \begin{itemize}
        \item 该模型假设目标出现的过程会出现多个阶段
        \item 仅在第一阶段出现的目标是给定的,之后阶段出现的新目标以概率分布的形式给出
        \item 与look-shoot-look模型最大的不同即为look-shoot-look不会出现新的目标
        \item 由于模型的复杂度较高,目前还没有比较好的精确解法,主要采用启发式解法
    \end{itemize}
\end{frame}

\subsection{Recent Developments}
\begin{frame}
    \frametitle{Sensor WTA problems}
    \begin{itemize}
        \item 背景:在现代化战争中,需要引入传感器对要打击的目标进行识别,类似地还需要引入指挥所等要素
        \item 下面是一个较为简单的模型:$S$表示传感器的集合,$o$表示传感器的数目
    \end{itemize}
    \begin{align*} \tag{Sensor}
        \min\quad & \sum_{j=1}^n V_j \left( \prod_{i=1}^m {\prod_{s = 1}^{o} (1 -  p_{isj})^{x_{isj}}} \right) \\ 
        \mathrm{s. t.}\quad &\sum_{j=1}^n {\sum_{s=1}^o x_{isj}} \leq 1\quad \forall ~i \in I,\\
        &\sum_{j=1}^n {\sum_{i=1}^m x_{isj}} \leq 1\quad \forall ~s \in S,\\
        & x_{isj} \in \left\{ 0,1 \right\} \quad \forall~ j\in J , ~ i \in I,~ s\in S.
    \end{align*}
    \begin{itemize}
        \item 两个约束分别是每个武器与每个传感器仅能分配给一个目标。
    \end{itemize}
\end{frame}

\begin{frame}
    \frametitle{Multi-objective programs}
    \begin{itemize}
        \item 在真实的作战环境中,不可能仅考虑对目标的杀伤概率
        \item 比较常见引入的目标包括最小化花费、最小化作战时间
        \item 最小化花费:对目标$j$使用武器$i$需要一个固定的花费
        \item 最小化作战时间:使用武器攻击目标需要耗费时间,为了达到作战目的也需要最小化作战时间
        \item 多个目标的结合方式包括将多个目标组合在一起,或者同时优化三个不同的目标,希望找到所有非占优解。在一些论文中,还提出了算法与指挥官结合的方法,即在算法执行过程中由指挥官选择不同目标的重要性。
    \end{itemize}
\end{frame}



\section{Outer Approximation}
\begin{frame}
    \frametitle{basic formulation}
    \begin{columns}
        \begin{column}{.70\linewidth}
            \footnotesize
            \begin{itemize}
                \item $I = \left\{1,\cdots,m\right\} $, 武器集合.
                \item $J = \left\{1,\cdots,n\right\} $, 目标集合.
                \item $p_{ij}\in [0,1]$, $i$击落$j$的概率.
            \end{itemize}
        \end{column}
    \hspace{-1cm}
        \begin{column}{.35\linewidth}
            \footnotesize
            \begin{itemize}
                \item $a_j$, 目标$j$的权重.
                \item $x_{ij}$,武器$i$是否攻击$j$.
            \end{itemize}
        \end{column}
    \end{columns}
    
    \begin{align*} \tag{S0}
        \min\quad & \sum_{j=1}^n a_j \left( \prod_{i=1}^m (1 -  p_{ij})^{x_{ij}} \right) \\ 
        \mathrm{s. t.}\quad &\sum_{j=1}^n x_{ij} \leq 1\quad \forall ~i \in I,\\
        & x_{ij} \in \left\{ 0,1 \right\} \quad \forall~ j\in J , ~ i \in I.
    \end{align*}
\end{frame}

\begin{frame}
    \frametitle{formulation}
    \begin{equation*}
        \prod_{i=1}^{m}{(1-p_{ij}x_{ij})} \iff \prod_{i=1}^{m}{(1-p_{ij})^{x_{ij}}}, x \in \left\{ 0,1 \right\}
    \end{equation*}
    \begin{itemize}
        \item 当两个问题限定$x$取值为$0,1$时,取值相等
    \end{itemize}
\end{frame}

\begin{frame}
    \frametitle{目标函数为多个凸函数的加权和}
    采用后一种形式:
    \begin{equation*}
        f_j(x) =  \prod_{i=1}^m (1 -  p_{ij})^{x_{ij}}\quad \forall~  j \in J
    \end{equation*}
    考虑该问题的Hessian矩阵$H$:
    \begin{equation*}
        \frac{\partial f_j(x)}{\partial x_{aj} \partial x_{bj}} = \ln(1 - p_{aj}) \ln(1 - p_{bj}) f_j(x)
    \end{equation*}
    {
    \scriptsize
    \begin{equation*}
        H = f(x)\begin{bmatrix}
            \ln(1 - p_{1j}) \ln(1 - p_{1j}) & \ln(1 - p_{1j}) \ln(1 - p_{2j}) &\cdots & \ln(1 - p_{1j}) \ln(1 - p_{mj}) \\
            \ln(1 - p_{2j}) \ln(1 - p_{1j}) & \ln(1 - p_{2j}) \ln(1 - p_{2j}) &\cdots & \ln(1 - p_{2j}) \ln(1 - p_{mj}) \\
            \vdots & \vdots &\ddots &\vdots \\
            \ln(1 - p_{mj}) \ln(1 - p_{1j}) & \ln(1 - p_{mj}) \ln(1 - p_{2j}) &\cdots & \ln(1 - p_{mj}) \ln(1 - p_{mj})


        \end{bmatrix}
    \end{equation*}
    }
\end{frame}

\begin{frame}
    \frametitle{目标函数为多个凸函数的加权和}
    若记
    \begin{equation*}
        l = \begin{bmatrix}
            \ln(1 - p_{1j}) & \ln(1 - p_{2j}) & \cdots & \ln(1 - p_{mj})
        \end{bmatrix}
    \end{equation*}
    可得到Hassien矩阵的形式为:
    \begin{equation*}
        H = f(x)l\cdot l^T
    \end{equation*}
    可以看出该矩阵是个秩一矩阵,即各列之间是线性相关的,从而其行列式为0,从而得到原函数是凸的。
\end{frame}

\begin{frame}
    \frametitle{transformed model}
    原模型的目标函数为:
    \begin{equation*}
        \sum_{j=1}^n a_j \left( \prod_{i=1}^m (1 -  p_{ij})^{x_{ij}} \right)
    \end{equation*}
    {\footnotesize
    希望通过引入辅助变量的形式,将非线性项从目标函数转化到约束中。
    
    若记$\eta_j$ 为代表 $\prod_{i=1}^m (1 -  p_{ij})^{x_{ij}}$引入的变量,原问题可转化为:
    \begin{align*} \tag{S0'}
        \min\quad & \sum_{j=1}^n a_j \eta_j \\ 
        \mathrm{s. t.}\quad & \eta_j \geq \prod_{i=1}^m (1 -  p_{ij})^{x_{ij}}, \quad \forall ~ j \in J \\ 
        &\sum_{j=1}^n x_{ij} \leq 1\quad \forall ~ i \in I,\\
        & x_{ij} \in \left\{ 0,1 \right\} \quad \forall ~ j\in J\quad i \in I.
    \end{align*}
    }
\end{frame}

\begin{frame}
    \frametitle{Basic idea of outer approximation}
    若$f(x)$为凸函数,则对于可行域中的任意一个点$x^*$,有
    \begin{equation*}
        f(x) \geq f(x^*) + \nabla f(x^*)(x - x^*)
    \end{equation*}
    因此若原问题的约束为$\eta \geq f(x)$,则此时
    \begin{equation*}
        \eta \geq f(x) \geq f(x^*) + \nabla f(x^*)(x - x^*)
    \end{equation*}
    一定成立。
    \begin{itemize}
        \item 即对于任意一个给定的点,都可以引入一个线性约束,保证可行域满足该约束。
        \item 外逼近方法即希望利用整数特性,通过引入多个线性约束来替代某个非线性约束,但保证转化前后问题是等价的。
    \end{itemize}
    
    
\end{frame}

\begin{frame}
    \frametitle{constraint formulation of outer approximation}
    仍记$f_j(x) =  \prod_{i=1}^m (1 -  p_{ij})^{x_{ij}}\quad j \in J$,则对于任意给定的可行域中的点$\bar{x}$,有
    \begin{align*}
        \nabla f(\bar{x})(x - \bar{x}) & = f(\bar{x})\sum_{i = 1}^m \ln(1-p_{ij})(x_{ij} - \bar{x_{ij}})\\
        &= f(\bar{x})\sum_{i = 1}^m \ln(1-p_{ij})x_{ij} - f(\bar{x})\sum_{i = 1}^m \ln(1-p_{ij})\bar{x_{ij}}
    \end{align*}
    其中前一项是变量的线性组合,有一项在$\bar{x}$给定时是常数。

    依据此外逼近约束的形式,即可得到问题的外逼近模型:
\end{frame}

\begin{frame}
    \frametitle{outer approximation model}
    
    {\scriptsize
    若记问题中的全部整数可行解为集合$X$,则模型可以描述为
    \begin{align*} \tag{OA}
        \min\quad & \sum_{j=1}^n a_j \eta_j \\ 
        \mathrm{s. t.}\quad & \eta_j \geq f(\bar{x})\sum_{i = 1}^m \ln(1-p_{ij})(x_{ij} - \bar{x_{ij}}) + f(\bar{x}), \quad \forall ~ j \in J,\ \bar{x} \in X \\ 
        &\sum_{i=1}^m x_{ij} \leq 1\quad \forall ~ j \in J,\\
        & x_{ij} \in \left\{ 0,1 \right\} \quad \forall ~ j\in J\quad i \in I.
    \end{align*}
    可以看出,此时外逼近约束的个数非常多,对问题的求解造成了一定的困难,因此考虑限制后的模型,即近考虑部分约束。
    }
   
\end{frame}

\begin{frame}
    \frametitle{Idea of outer approximation method}
    {
    \scriptsize
    基本思路:外逼近约束个数过多,仅考虑其子集$\hat{X} \subseteq X$
    \begin{align*}
        \min\quad & \sum_{j=1}^n a_j \eta_j \\ 
        \mathrm{s. t.}\quad & \eta_j \geq f(\bar{x})\sum_{i = 1}^m \ln(1-p_{ij})(x_{ij} - \bar{x_{ij}}) + f(\bar{x}), \quad j \in J,\ \bar{x} \in X \\ 
        &\sum_{i=1}^m x_{ij} \leq 1\quad j \in J,\quad x_{ij} \in \left\{ 0,1 \right\} \quad j\in J\quad i \in I
    \end{align*}
    \begin{enumerate}
        \item 去除全部外逼近约束,给出当前最优解$x, \eta$
        \item 检验当前最优解是否能满足全部的外逼近约束,若可以,迭代终止,执行结束
        \item 否则选择违背最严重的外逼近约束,加入到其中,重新求解并执行步骤2
    \end{enumerate}
    }
\end{frame}

\begin{frame}
    \frametitle{choose violate constraint}
    {
    \scriptsize
    \begin{align*}
        \min\quad & \sum_{j=1}^n a_j \eta_j \\ 
        \mathrm{s. t.}\quad & \eta_j \geq f(\bar{x})\sum_{i = 1}^m \ln(1-p_{ij})(x_{ij} - \bar{x_{ij}}) + f(\bar{x}), \quad j \in J,\ \bar{x} \in X \\ 
        &\sum_{i=1}^m x_{ij} \leq 1\quad j \in J,\quad x_{ij} \in \left\{ 0,1 \right\} \quad j\in J\quad i \in I
    \end{align*}
    \begin{itemize}
        \item 除非已经找到了最优解,否则限制后的问题的解一定会带来一个违背的约束(放大了可行域,目标函数一定会更小,说明它不在原问题的可行域当中)
        \item 可以直观地理解,违背即指$\eta_j$在可行域之外了,由于本身是求的最小化的问题,因此在找到的点加割即为最违背的割。
        \item 由此可以得到当求解出一个在可行域当中的点时,如果$\eta < f_j(x)$,就说明这个点应该被包含在可行域之外,因此可以在这个点加入一个割。
    \end{itemize}
    }
\end{frame}


\section{Columnn Generation}
\begin{frame}
    \frametitle{Basic Idea}
    \begin{itemize}
        \item 许多线性规划问题的列数(变量)太多而无法明确考虑所有变量
        \item 在算法的开始仅使用部分变量,并假设其他变量全部为0
        \item 迭代地将有可能改进目标函数的变量添加到数学规划模型中。一旦可以证明添加新变量将无法再提高目标函数的值,终止迭代过程并得到最优解。
    \end{itemize}
\end{frame}

\begin{frame}
    \frametitle{How to use column generation in WTA Problem}
    \begin{itemize}
		\item \textbf{基本思路}:通过将所有目标场景$S$列出来的方式,将问题转化为一个线性规划问题。
		\item \textbf{转化方式}:假设一共有$m$个武器,则对于任意一个目标$j$,每个武器可以选择打击$j$或不打击$j$,因此共有$2^m$个不同的打击方案。
		\item \textbf{一个例子}:一共有8个武器,则使用第1,3,6号武器的攻击方案即记为$s_{[1,0,1,0,0,1,0,0]}$,$|S| = 2^8 = 256$
		\item $n_{si}$:0-1变量,表示在第$s$个场景下,是否启用武器$i$. 在上面的例子中,$n_{s1} = 1,\ n_{s2} = 0$
		\item $q_{js} = a_j \prod_{i = 1}^m (1 - n_{si}\cdot p_{ij})$:使用第方案$s$打击目标$j$的概率并加权,比如$q_{3s}$就是采用1,3,6号武器打击$3$号目标的击毁概率乘以目标$j$的权重。
	\end{itemize}
\end{frame}

\begin{frame}
    \frametitle{transformed formulation}
    \optimalProblem{\min}{\sumFromTo{j = 1}{n}{\sumFromTo{s = 0}{2^m -1}{q_{js}y_{js}}}\tag{CG}}{\sumFromTo{j = 1}{n}{\sumFromTo{s = 0}{2^m -1}{n_{si}y_{js}}}\leq 1 \quad &\forall ~ i \in I\\& \sumFromTo{s = 0}{2^m - 1}{y_{js}} = 1 \quad &\forall ~ j \in J\\& y_{js} \in \left\{ 0,1 \right\} &\forall ~ j\in J,\ s\in S}
    \begin{columns}
        \begin{column}{.70\linewidth}
            \footnotesize
            \begin{itemize}
                \item $I = \left\{1,\cdots,m\right\} $, 武器集合.
                \item $J = \left\{1,\cdots,n\right\} $, 目标集合.
                \item $S = \left\{1,\cdots,2^m\right\}$, 场景集合.
            \end{itemize}
        \end{column}
    \hspace{-3cm}
        \begin{column}{.70\linewidth}
            \footnotesize
            \begin{itemize}
                \item $n_{si}$:场景$s$是否使用武器$i$
                \item $y_{js}$:0-1变量,是否对目标$j$应用$s$
                \item $q_{js}$:赋权后的场景$s$对$j$的摧毁概率
            \end{itemize}
        \end{column}
    \end{columns}
\end{frame}

\begin{frame}
    \frametitle{transformed formulation continuous}
    \optimalProblem{\min}{\sumFromTo{j = 1}{n}{\sumFromTo{s = 0}{2^m -1}{q_{js}y_{js}}}\tag{CG}}{\sumFromTo{j = 1}{n}{\sumFromTo{s = 0}{2^m -1}{n_{si}y_{js}}}\leq 1 \quad &\forall ~ i \in I\\& \sumFromTo{s = 0}{2^m - 1}{y_{js}} = 1 \quad &\forall ~ j \in J\\& y_{js} \in \left\{ 0,1 \right\} & \forall ~ j\in J,\ s\in S}
    \begin{itemize}
        \item \textbf{目标函数}:最小化赋权后的消灭概率
        \item \textbf{第一个约束}:每个武器最多只能攻击一个目标
        \item \textbf{第二个约束}:每个目标恰好安排一个场景
    \end{itemize}
\end{frame}
\begin{frame}
    \frametitle{column enumeration}
    \optimalProblem{\min}{\sumFromTo{j = 1}{n}{\sumFromTo{s = 0}{2^m -1}{q_{js}y_{js}}}\tag{CG}}{\sumFromTo{j = 1}{n}{\sumFromTo{s = 0}{2^m -1}{n_{si}y_{js}}}\leq 1 \quad &\forall ~ i \in I\\& \sumFromTo{s = 0}{2^m - 1}{y_{js}} = 1 \quad &\forall ~ j \in J\\& y_{js} \in \left\{ 0,1 \right\} & \forall ~ j\in J,\ s\in S}

    \begin{itemize}
        \item 列枚举方法的基本思路是采用较好的方式对全部列进行枚举
        \item 文章中采用了两个技巧:weapon number bounding and weapon domination
        \item weapon number bounding:如果给一个目标分配了太少或者太多武器的场景,一定会导致无法改进。
        \item weapon domination:针对某个目标,安排某个武器一定比安排另一个武器更好
    \end{itemize}
\end{frame}

\begin{frame}
    \frametitle{LP Relaxition}
    \optimalProblem{\min}{\sumFromTo{j = 1}{n}{\sumFromTo{s = 0}{2^m -1}{q_{js}y_{js}}}\tag{CG-LP}}{\sumFromTo{j = 1}{n}{\sumFromTo{s = 0}{2^m -1}{n_{si}y_{js}}}\leq 1 \quad &\forall ~ i \in I\\& \sumFromTo{s = 0}{2^m - 1}{y_{js}} = 1 \quad &\forall ~  j \in J\\& y_{js} \geq 0 &\forall ~ j\in J,\ s\in S}
    \begin{itemize}
        \item 将问题进行线性松弛,第三个约束可以直接放成$y_{js} \geq 0$
        \item 此时线性规划的特点为包含了$n \times 2^m$列
        \item 列生成思路:仅取其中的部分列(变量),因为在线性规划的最优解中,最多有$m+n$个变量不为0,即其他变量对应的对偶约束不积极。
    \end{itemize}
\end{frame}

\begin{frame}
    \frametitle{Dual Problem}
    
    \optimalProblem{\max}{\sumFromTo{i=1}{m}{u_i} + \sumFromTo{j=1}{n}{v_j}\tag{CG-Dual}}{\sumFromTo{i=1}{m}{x_{is}u_i}+ v_j \leq q_{js}\quad \forall ~(s,j) \in X\\ &u_i \leq 0,\quad v_j\ \mathrm{free}}
    \begin{itemize}
        \item 若记$X = \left\{(s,j) | s\in S , j \in J \right\}$,每一个$X$中的元素都对应一个约束。
        \item \item 选择原问题中的部分列相当于对偶问题选择了部分行。
    \end{itemize}
\end{frame}

\begin{frame}
    \frametitle{Restricted Dual Problem}
    
    \optimalProblem{\max}{\sumFromTo{i=1}{m}{u_i} + \sumFromTo{j=1}{n}{v_j}\tag{CG-Dual-R}}{\sumFromTo{i=1}{m}{x_{is}u_i}+ v_j \leq q_{js}\quad ~\forall (s,j) \in \hat{X} \\ &u_i \leq 0,\quad v_j\ \mathrm{free}}
    \begin{itemize}
        
        \item 选择部分约束即为仅考虑由部分$(s,j)$生成的约束,记$\hat{X} \subseteq X$
        \item 求解得到$u^*, v^*$并带入原对偶,若都成立则一定为最优解。
        \item 否则希望选择一个违背最严重的,即$\sumFromTo{i=1}{m}{x_{is}u_i}+ v_j > q_{js}$中最大的。
    \end{itemize}
\end{frame}

\begin{frame}
    \frametitle{subproblem}
    \optimalProblem{\min}{a_j \prod_{i=1}^{m}{(1-p_{ij}x_{is})} -\sumFromTo{i=1}{m}{x_{is}}u_i - v_j}{j \in J,\quad s \in S\tag{CG-sub}}
    \begin{itemize}
        \item 希望选择一个违背最严重的,即$\sumFromTo{i=1}{m}{x_{is}u_i}+ v_j > q_{js}$中最大的。
        \item 将$q_{js}$的定义带入,即可得到上述子问题。
        \item 可以将问题分离,之后对每个给定的目标$j$求解,其直观为使用每个武器攻击需要一个花费,希望平衡开启武器的花费和消灭目标的概率。
        \item 该子问题的形式的形式可以有效利用之前的外逼近方法。
    \end{itemize}
\end{frame}



\section{Future Research Plan}
\begin{frame}
    \frametitle{Future Research Plan}
    \begin{itemize}
        \item 实现列生成方法
        \item 在问题中加入传感器
    \end{itemize}
\end{frame}

\begin{frame}
    \frametitle{Alternate applications}
    \begin{align}
        \min\quad & \sum_{j=1}^n V_j \left( \prod_{i=1}^m (1 -  p_{ij})^{x_{ij}} \right) \\ 
        \mathrm{s. t.}\quad &\sum_{j=1}^n x_{ij} \leq 1\quad \forall ~i \in I,\\
        & x_{ij} \in \left\{ 0,1 \right\} \quad \forall~ j\in J , ~ i \in I.
    \end{align}
    \begin{itemize}
        \item 该模型不仅可以被应用在武器目标分配问题中,一些资源有限的分配问题都可以利用该模型。
        \item 目标函数可以变化为$\sum_{j=1}^n V_j f_j(x)$,只要它满足一定的条件,比如是凸函数,则可以考虑使用上述两个方法来进行求解。
    \end{itemize}
\end{frame}

\section{Acknowledgement and Reference}
\begin{frame}
\textcolor{myNewColorA}{\Huge{\centerline{Thank you!}}}
\end{frame}

\begin{frame}
{\small
\bibliographystyle{abbrvnat}
\bibliography{reference}
}
\end{frame}


\end{document}